\section{Anzeigen von benachbarten Flächenstücken (Dreiecken)}
\begin{center}
 % 14.1
\end{center}
Gehört der Rand eines Dreiecks zum Dreieck?\\
Zu welchem Dreieck?
\paragraph*{Gesucht} Eine konsistente Regel, die bei mehreren aneienanderstoßenden Flächen feslegt, zu welcher Fläche
jeder Bildpunkt gehört.
\paragraph*{Eine Möglichkeit} Ein Pixel, das auf dem Rand liegt wird (in Gedanken) horizontal nach rechts verschoben.
	Es gehört zu der Fläche, wo es dann landet. Wenn es dabei auf einer horizontalen Kante liegt, dann wird es zur
	oberen Fläche gerechnet.
\begin{center}
 %14.2
\end{center}
\paragraph*{Vorraussetzung} Exakte Arithmetik

\section{Lineare Interpolation in Weltkoordinaten}
Interpolation von Helligkeitswerden (Gouraud-Schattierung), von Normalenrichtungen (Phong-Schattierung), bei Texturen
immer in \emph{Weltkoordinaten}! Nicht NDC.
\paragraph{Gegeben} Eine Strecke $P_1P_2$ in NDC (oder Bildschirmkoordinaten) als Bild einer Strecke $P_1P_2$
	in Weltkoordinaten.

	An den Endpunkten sind Werte $I_1, I_2$ gegeben

	(Intensitätswerte $I_1^R, I_1^G, I_1^B,...$, oder
	Normalenrichtungen $N_1^x, N_1^y, N_1^z,...$)

	Wir wollen diese Werte für die dazwischen liegenden Punkte der Strecke in Weltkoordinaten interpolieren.

	\hrulefill

	Wir kennen die Transformationsmatrix $M = M^{\mathrm{Welt, NDC}}$:
	\[ M \vfour{x_{\mathrm{Welt}}}{y_{\mathrm{Welt}}}{z_{\mathrm{Welt}}}{w_{\mathrm{Welt}}}
		= \vfour{x_{\mathrm{NDC}}}{y_{\mathrm{NDC}}}{z_{\mathrm{NDC}}}{w_{\mathrm{NDC}}}
	\qquad\qquad P_1 = \vfour{x_1^{\mathrm{NDC}}}{y_1^{\mathrm{NDC}}}{z_1^{\mathrm{NDC}}}{1}
	\qquad P_2 = \vfour{x_2^{\mathrm{NDC}}}{y_2^{\mathrm{NDC}}}{z_2^{\mathrm{NDC}}}{1}\]
	\[P_1\text{ in homogenen Weltkoordinaten} = M^{-1} \vfour{x_1^{\mathrm{NDC}}}{y_1^{\mathrm{NDC}}}{z_1^{\mathrm{NDC}}}{1}
	 = \vfour{x_1^{\mathrm{Welt}}}{y_1^{\mathrm{Welt}}}{z_1^{\mathrm{Welt}}}{w_1^{\mathrm{Welt}}}
	 \hat = \vthree{x_1^{\mathrm{Welt}}/w_1^{\mathrm{Welt}}}{y_1^{\mathrm{Welt}}/w_1^{\mathrm{Welt}}}
		{z_1^{\mathrm{Welt}}/w_1^{\mathrm{Welt}}}\text{ in kartesischen Weltkoordinaten}
	\]
\paragraph{Annahme:} $I: \mathbb{R}^3 \to \mathbb{R}$ linear $I(P_1) = I_1 \qquad I(P_2) = I_2$

\hrulefill

Betrachte eine "`homogene Erweiterung"': $\hat I: \mathbb{R}^4 \to \mathbb{R}$ von I
\[\hat I\left(\vfour{\lambda x}{\lambda y}{\lambda z}{\lambda w}\right) =
	\lambda \cdot \hat I\left(\vfour{x}{y}{z}{w}\right)
	\qquad
	\hat I\left(\vfour{x}{y}{z}{1}\right) = I\left(\vthree{x}{y}{z}\right)\]
\begin{enumerate}
 \item Die Funktion $\hat I$ ist linear;
 \item Aus $\hat I$ kann man $I$ ausrechnen:
	\[I\left(\text{Punkt $\vfour{x}{y}{z}{w}$ in homogenen Koordinaten}\right) = \frac{1}{w} \cdot
		\hat I\left(\vfour{x}{y}{z}{w}\right)\]
\end{enumerate}
\begin{align*}
 I\left(\text{Punkt in homogenen Koordinaten }\vfour{x}{y}{z}{w}\right) &= \\
 I\left(\text{Punkt in homogenen Koordinaten }\vfour{x/w}{y/w}{z/w}{1}\right) &=
	\hat I \left(\vfour{x/w}{y/w}{z/w}{1}\right) = \frac{1}{w} \cdot \hat I\left(\vfour{x}{y}{z}{w}\right)\\
 I\left(\text{Punkt in karthesischen Koordinaten }\vthree{x/w}{y/w}{z/w}\right) &=
\end{align*}

\begin{enumerate}
 \item Berechne $P_1$ und $P_2$ in homogenen Weltkoordinaten:
	\[\vfour{x_1^{\mathrm{Welt}}}{y_1^{\mathrm{Welt}}}{z_1^{\mathrm{Welt}}}{w_1^{\mathrm{Welt}}}
	= M^{-1} \cdot \vfour{x_1^{\mathrm{NDC}}}{y_1^{\mathrm{NDC}}}{z_1^{\mathrm{NDC}}}{1}
	\text{ und } \vfour{x_2^{\mathrm{Welt}}}{y_2^{\mathrm{Welt}}}{z_2^{\mathrm{Welt}}}{w_2^{\mathrm{Welt}}}
	= M^{-1} \cdot \vfour{x_2^{\mathrm{NDC}}}{y_2^{\mathrm{NDC}}}{z_2^{\mathrm{NDC}}}{1}\]
 \item Berechne $\hat I_1$ und $\hat I_2$ an diesem Punkt in $\mathbb{R}^4$
	\begin{align*}
	 I(P_1) =\ &I_1 = \frac{1}{w} \cdot \hat I_1\\
		&\boxed{\hat I_1 = w_1^{\mathrm{Welt}} \cdot I_1,\ \hat I_2 = w_2^{\mathrm{Welt}} \cdot I_2}
	\end{align*}
 \item Lineare Interpolation $P(\lambda) = \lambda_2 + (1 - \lambda) P_1$ in NDC
	\paragraph{Gesucht} $I(\lambda) = I(P(\lambda))$
		\[\vfour{x(\lambda)}{y(\lambda)}{z(\lambda)}{1} = \lambda \vfour{x_2}{y_2}{z_2}{1}
			+ (1-\lambda) \vfour{x_1}{y_1}{z_1}{1} \]
		\[\boxed{\hat I = \lambda \cdot \hat I_2 + (1-\lambda) \hat I_1}\]
		\[\boxed{w^{\mathrm{Welt}}(\lambda) = \lambda \cdot w^{\mathrm{Welt}}_2 + (1-\lambda) w^{\mathrm{Welt}}_1}
		\qquad I(\lambda) = \frac{\hat I(\lambda)}{w^{\mathrm{Welt}}(\lambda)}\]
\end{enumerate}
\paragraph*{Zusammenfassung} $w_1^{\mathrm{Welt}}, w_2^{\mathrm{Welt}} = $ letzte Komponente von $M^{-1} \vthree{\ }{\cdots}{\ }$
	\[\boxed{I(\lambda)} = \frac{\overbrace{w_2^{\mathrm{Welt}} \cdot I_2}^{\hat I_2} \cdot \lambda
		+ \overbrace{w_1^{\mathrm{Welt}} \cdot I_1}^{\hat I_1} \cdot (1-\lambda)}
					{w_2^{\mathrm{Welt}} \lambda + w_1^{\mathrm{Welt}} (1 - \lambda )}\]

\paragraph{Einbau in Füllalgorithmus}
 \rnode{zus}{zusätzlich}

\begin{lstlisting}[mathescape=true,backgroundcolor=\color{white}]
FuelleGerade($x_1$, $x_2$,$y$,$z_1$,$z_2$,$\rnode{w1}{w_1}$,$\rnode{w2}{w_2}$,$I_1$,$I_2$)
	$\hat I_1 = w_1 I_1$
	$\hat I_2 = w_2 I_2$
	$\Delta \hat I = ...$
	$\Delta w = ...$
	$\vdots$
	while $x < x_2$
			Setpixel($x$,$y$,...,$I$)
			$x := x+1$
			$z := z + \Delta z$
			$\hat I := \hat I + \Delta I$	// Zaehler
			$w := w + \Delta w$		// Nenner
			$I = \tfrac{\hat I}{w}$
\end{lstlisting}
\ncline{->}{zus}{w1}
\ncline{->}{zus}{w2}
analog für mehrere Größen $I^R, I^G, I^B$ nimmt man $\hat I^R$, $\hat I^G$, $\hat I^B$, 
$w$ und $\Delta w$ braucht man nur einmal.

\paragraph{Bemerkung} Die $z$-Koordinate kann man direkt in NDC interpolieren

\section{Transformation von Normalvektoren bei affinen Transformationen}
z. B. Objektkoordinaten auf Weltkoordinaten
\begin{center}
 % 14.4
\end{center}
\[T = \begin{pmatrix}
       \nicefrac{b}{2} & 0 & 0 & g_x\\
       0 & \nicefrac{b}{2} & 0 & g_y\\
       0 & 0 & h & g_z\\
       0 & 0 & 0 & 1
      \end{pmatrix}
\]
.... (s. Foto?)\\
Bei einer Affinen Transformation mit $3 \times 3$-Transformationsmatrix $A$ muss man Nornalvektoren nicht
mit $A$, sondern von links nach rechts mit $(A^{-1})^T$ multiplizieren und anschließend \emph{normieren}.
(Vektoren als Spalten betrachten)

Fals $A$ orthogonal ist, dann ist $A^T = A^{-1}$, $(A^{-1})^T = A$



