\section{"`rendering pipeline"' -- vom geometrichen Modell zum Rasterbild}
\begin{psmatrix}
 % Kästen mit Pfeilen verbinden
 \frame{
	\begin{minipage}{3cm}
		Durchlaufen des geometrischen Modells (Menge der Fläche)
	\end{minipage}
 } &
 \frame{
	\begin{minipage}{3cm}
		Transformation in Weltkoordinaten
	\end{minipage}
 } &
 \frame{
	\begin{minipage}{3cm}
		Beleuchtung (Licht und Schatten)
	\end{minipage}
 } & % Pfeillabel: Kameraposition
 \frame{
	\begin{minipage}{3cm}
		Transformation in normalisierte Gerätekoordinaten
	\end{minipage}
 } \\
 \frame{
	\begin{minipage}{3cm}
		Abschneiden von Ob"-jekten, die außer"-halb des sicht"-baren Kegel"-stumpfes lie"-gen (clipping)
	\end{minipage}
 } &
 \frame{
	\begin{minipage}{3cm}
		Sichtbarkeit (Elimination verdeckter Objekte)
	\end{minipage}
 } & %<- vertauschbar ->
 \frame{
	\begin{minipage}{3cm}
		Rastern
	\end{minipage}
 } &
 \begin{minipage}{3cm}
	Rasterbild
	\begin{center}
	 \psset{unit=0.1cm}
	 \begin{pspicture}(25,25)
	  \psgrid[gridlabels=0pt,gridcolor=gray]
	  \psdot(24,24)
	 \end{pspicture}
	\end{center}
 \end{minipage}
 \ncline{->}{1,1}{1,2}
 \ncline{->}{1,2}{1,3}
 \ncline{->}{1,3}{1,4}\naput[nrot=90]{
	 \begin{minipage}{1.5cm}
	  Ka"-me"-ra"-po"-sit"-ion
	 \end{minipage}}
 \nccurve[angleA=-90,angleB=90,ncurvA=0.2,ncurvB=0.2]{->}{1,4}{2,1}
 \ncline{->}{2,1}{2,2}
 \ncline{->}{2,2}{2,3}
 \ncline{->}{2,3}{2,4}
 \nccurve[angleA=40,angleB=100,linestyle=dashed]{<->}{2,2}{2,3}
\end{psmatrix}\\
nur \emph{eine} mögliche Organisation; andere Reihenfolgen sind möglich