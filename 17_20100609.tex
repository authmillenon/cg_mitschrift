\section{Darstellung von geometrischen Modellen in Grafikbibliotheken}
\begin{center}
 17.1
\end{center}
typische Funktionsaufrufe in einer Grafikbibliothek wie OpenGL
\begin{lstlisting}
startPolygon();
	setzeDiffuseFarbe($r$,$g$,$b$);
	// ... weitere Materialeigenschaften
	Knoten($x_1,y_1,z_1$);
	Knoten($x_2,y_2,z_2$);
	Knoten($x_3,y_3,z_3$);
endPolygon();
startPolygon();
	Knoten($x_1,y_1,z_1$);
	Knoten($x_6,y_6,z_6$);
	Knoten($x_5,y_5,z_5$);
	Knoten($x_3,y_3,z_3$);
	Knoten($x_2,y_2,z_2$);
endPolygon();
// ...
\end{lstlisting}
\begin{center}
Modellkoordiaten $\rnode{to}\rightarrow$ Weltkoordinaten {\color{gray}$\rightarrow ...$}\\[2em]
\rnode{CT}{CT = "`current transformation matrix"'}\\
aktuelle Transformationsmatrix
\ncline{->}{CT}{to}
\end{center}
\paragraph*{typischer Ablauf: }
\begin{itemize}
\item Am Anfang wird auf CT die Identität initialisiert.
\item eine Folge von geometrischen Transformationen; die entsprechenden Transformationen werden \emph{von rechts}
	an CT dranmultipliziert.
\item Dazwischen werden die aktuellen Werte von CT auf einem Stapel gespeichert (Push) bzw. wiederhergestellt (Pop)
\end{itemize}
\begin{center}
 17.2
\end{center}
\begin{lstlisting}
setze af Einheitsmatrix				// CT = $I$
	multiply($T_3$)				// CT = $T_3$
		multiply($T_2$)			// CT = $T_3 \cdot T_2$
		Beschreibe Turm
			push
				multiply($T_1$)	// CT = $T_3 \cdot T_2 \cdot T_1$
				beschreibe Zinne...
			pop
			push
				multiply($T_1$)	// CT = $T_3 \cdot T_2 \cdot T_1'$
				beschreibe Zinne...
			pop
			$\vdots$
\end{lstlisting}

\section{Constructive Solid Geometry (CSG)}
\subsection{Grundlagen}
Objekte werden als gefülltes Volumen betrachten, nicht als Randflächen.
\begin{itemize}
 \item \textbf{Grundkörper:} Würfel, Zylinder; konvexe Polyeder
 \item \textbf{Boolesche Operationen:} $\cup$, $\cap$, $\setminus$ (eigentlich Mengenoperationen)
	z. B:
	\begin{center}
	17.3
	\end{center}
	\begin{center}
	17.4
	\end{center}
	Man möchte nur reguläre Mengen betrachten. Eine Menge $K$ ist regulär, wenn sie der Abschluss ihres Inneren ist:
	\[K = \overline{K^\circ}\]
	\paragraph*{zur Erinnerung:} Das Innere der Menge $A$ ist definiert mit
	\[A^\circ = \{x \in A\ |\ \exists\varepsilon > 0:\ \text{Kreis mit Radius $\varepsilon$ um $x$ $\subseteq A$}\}\]
	\begin{center}
	17.5
	\end{center}
	Der Abschluss von $A$ ist definiert mit
	\[\overline A = \mathbb{R}^2 \setminus (\mathbb{R}^2 - A)^\circ\]
	(bzw. mit $\mathbb{R^3}$ im Raum)
	\begin{center}
	17.6
	\end{center}
	Der Rand von $A$ ist definiert mit
	\[\partial A := \overline A - A^\circ\]
	$\Rightarrow$ regularisierte Boolesche Operationen
	\begin{align*}
	 A \cup^* B &:= \overline{(A \cup B)^\circ}\\
	 A \cap^* B &:= \overline{(A \cap B)^\circ}\\
	 A \setminus^* B &:= \overline{(A \setminus B)^\circ}
	\end{align*}
	\begin{center}
	 17.7
	\end{center}
\end{itemize}
\subsection{Darstellung in der Computergrafik}
\[A \cup^* \left[(B \cap^* C) \setminus^* (D \cap^* E)\right]\]
\begin{center}
 17.8
\end{center}
Umwandlung in eine expliziete Darstellung des Randes ist aufwändig.

Man mss Snitte zwischen den Grundobjekten bestimmen. Anschließend muss manberechnen, welche der ausgeschnittenen
Oberflächenstücke den Rand des Ergebnisses bilden.

\paragraph*{Schnitt mit einer Geraden} (z. B. einem Sichtstrahl beim Raytraycing) ist relativ einfach:
\begin{itemize}
 \item Schneide den Strahl mit allen Grundbausteinen.
	sortiere die Schnittpunkte, bestimme den ersten Schnittpunkt, wo der Strahl in den Körper eintritt.
\end{itemize}

\chapter{Strahlverfolgung (Raytracing)}
gut geeignet für spiegelnde und transparente Körper.
\begin{center}
 17.9
\end{center}
\begin{enumerate}[A)]
 \item \textbf{"`Forward raytracing"'} Schicke Lichtstrahlen (Lichtteilchen) von der Lichtquelle aus und
	verfolge ihren Weg.
	
	Die allermeisten Teilchen werden irgendwo absorbiert, bevor sie ins Auge gelangen.\\
	$\Rightarrow$ extrem aufwändig.
 \item \textbf{"`Backward raytracing"'}
	\begin{enumerate}[1.]
	\item Bestimme Helligkeit mit einem normalen Beleuchtungsmodell
	\item Verfolge Sichtstrahlen vom auge ausgehend zurück, solange sie nur spiegelnde oder transparenter
		Flächen treffen.
	\end{enumerate}
	\begin{center}
	 17.10
	\end{center}
	Bei Spiegelung kann man entweder den Strahl vor dem Auftreffen (Auge) oder nach dem Auftreffen (Objekt)
	an der Spiegelungseben spiegeln $\rightarrow$ aus dem geknickten Sichtstrahl wird eine Gerade.
\end{enumerate}

\section{Gesetzmäßigkeiten}
\subsection{Reflexion}
\begin{center}
 12.11
\end{center}
$\alpha = \beta$: Einfallswinkel = Ausfallswinkel

\subsection{Brechung}
\begin{center}
 12.12
\end{center}
Übergang zwischen zwei Medien $M_1$ und $M_2$, die verschiedene optische Dichte $D_1$ bzw. $D_2$ haben.
\[\frac{\cos \alpha}{\cos \beta} = \underbrace{\frac{D_2}{D_1}}_{\text{fest für 2 gegebene Materialien}}\]
Die Brechung tritt fast nie in Reinform auf. Es tritt meist immer Reflexion gleichzeitig auf\\
Wenn $D_1 > D_2$ ist, dann kann in der Formel $\cos \beta > 1$ herauskommen.\\
$\rightarrow$ Dann wird der Strahl reflektiert (Totalreflexion)
