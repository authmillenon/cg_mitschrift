\begin{align*}
 x &= Ax & A &\in \mathbb{R}^{n \times n} \qquad b \in \mathbb{R}^n
	 & x &\in \mathbb{R}^n\\
 x^{(0)}{,}\ & x^{(1)} = A x^{(0)} + b, x^{(2)} = Ax^{(1)} + b
\end{align*}
\textsc{Gauß-Seidel}-Verfahren (iteratives Verfahren zur Lösung des Gleichungssystems). Konvergiert unter gewissen
Bedingungen. z. B. wenn die Summe der Amsolutbeträge in jeder Zeile von $A < 1$ ist.

Bei der Radiosity-Gleichung ist dies fast der Fall: $\le 1$.\\

Falls alle $\rho_i < 1$ sind: $< 1$

\subsubsection{Zusammenfassung}
\begin{enumerate}
 \item Flächen in kleine Stücke zerlegen
	\begin{enumerate}[a)]
	 \item gleichförmiges Gitter
		\begin{center}
		 20.1a
		\end{center}
	 \item angepasst an den erwarteten Lichtaustausch.
		\begin{center}
		 20.2
		\end{center}
		Wo Flächen nah an anderen Flächen sind wird die Zerlegung feiner.
	 \item adaptiv.\\
		Man rechnet zuerst mit einer gröberen Zerlegung und verfeinert sie dort, wo das Ergebnis
		stark schwankt
	\end{enumerate}
 \item Formfaktoren bestimmen (am zeitaufwändigsten)\\
	Können auch mit Hilfe von Grafikpuffern berechnet werden.
	\[\int\limits_{p \in A_i} \int\limits_{q \in A_j} ...\]
	\begin{center}
	 20.2
	\end{center}
	Für ein festes $p$ ähnelt die Berechnung des Integranden der Grafikpufferberechnung mit Kamera $p$. Diese
	Berechnung wird von Grafik-Hardware unterstützt
 \item Gleichungssystem lösen
 \item mit den so bestimmten (diffusen) Helligkeiten kann man dann die Szene darstellen
\end{enumerate}

\chapter{Transparenz}
\begin{center}
 20.3
\end{center}
teildurchsichtige Medien. z. B. Glas mit einem Bild, mit Staub, Nebel, Wolken, Rauch.

\section{RGBA-Modell bzw. $\boldsymbol{(r, g, b, \alpha)}$}
Ein halbdurchsichtiges Bild wird durch eine vierte Größe $\alpha$ (zusätzlich zu $R, G, B$) für jeden Bildpunkt
\begin{align*}
 \alpha &= 1 & &\text{vollständig undurchsichtig}\\
 \alpha &= 0 & &\text{vollständig durchsichtig}\\
 \alpha &= 0{,3} & &\text{$30\%$ wird von diesem Bild beigesteuert, }\\
	&	& &\text{$70\%$ kommt von dem was dahinter liegt}
\end{align*}
z. B. Bild eines Mauszeigers (idealisiert)
\begin{center}
 20.4
\end{center}
in Wirklichkeit natürlich "`pixelig"', Übergänge zwischen den kanten mit Antialiasing geglättet:
\begin{center}
 20.5
\end{center}
\[
\left.
 \begin{array}{r}
  (r,g,b,\alpha)\\
 (\bar r, \bar g, \bar b) \text{(Hintergrund)}
  \end{array}\right\} \text{Ergebnis} = (\alpha \cdot r + (1 - \alpha)\cdot \bar r,
					\alpha \cdot g + (1 - \alpha)\cdot \bar g,
					\alpha \cdot b + (1 - \alpha)\cdot \bar b)
\]
\begin{center}
 12.6
\end{center}
\rnode{Erg}{Ergebis} $= \alpha^1\,rgb^1 + (1 - \alpha^1)\,\alpha^2rgb^2 + (1-\alpha^1)\,(1-\alpha^2)\,\alpha^3\,rgb^3
	+ (1-\alpha^1)\,(1-\alpha^2)\,(1-\alpha^3)\,\rnode{rgb}{\underbrace{rgb}}$
	\begin{center}
	 \rnode{trans}{mehrere Transparente Schichten} \hspace{5cm} \rnode{vec}{Vektor aus 3 Komponenten}\\[1em]
	 20.7
	\end{center}
	\ncline{->}{trans}{Erg}\ncline{->}{vec}{rgb}
	\[\left(\underbrace{\frac{\alpha^1\,rgb^1 + (1 - \alpha)\,\alpha^2\,rgb^2}{\alpha^1 + (1-\alpha^1)\,\alpha^2}}_
		{rgb^\mathrm{neu}}, \underbrace{\alpha^1 + (1-\alpha^1)\,\alpha^2}_{\alpha^\mathrm{neu}}\right)\]
		Effekt auf eine dahinter liegende Fläche $\overline{rgb}$
		\[\alpha^\mathrm{neu} \, rgb^\mathrm{neu} + (1-\alpha^\mathrm{neu})\,\overline{rgb}\]
Wichtig: durchlässige Medien in der richtigen Reihenfolge kombinieren (z. B. vorne nach hintern, oder hinten nach vorne)

\section{Kombination mit Tiefenpuffer}
$(rgb^1, z^1, \alpha^1)\quad(rgb^2, z^2, \alpha^2)$ einer dieser Werte ist im Tiefenpuffer an einer bestimmten Stelle gespeichern,
	der andere soll dort hingeschrieben werden
\begin{align*}
 z^1 < z^2 &\ \text{(o. B. d. A.)}\\
 rgb^\mathrm{neu} &:= \frac{\alpha^1\,rgb^1 + (1-\alpha^1)\,\alpha^2\,rgb^2}{\alpha^1+(1-\alpha^1)\,\alpha^2}\\
 \alpha^\mathrm{neu} &:= \alpha^1 + (1-\alpha^1) \alpha^2\\
 z^\mathrm{neu} &:= \rnode{z}{z^1} 
\end{align*}
\rnode{not}{(eigentlich unzureichend}, aber notwendig, um wenigstens den undurchsichtigen Fall richtig zu behandeln)
\ncangle[angleA=90,angleB=-90]{->}{not}{z}
\\[1em]
Diese Methode funktioniert, wenn man höchstens \emph{ein} halbdurchlässiges Pixel $(0 < \alpha < 1)$ zeichen möchte
oder wenn \emph{alle} Objekte von vorne nach hinten oder von hinten nach vorne eingefügt werden.

\chapter{Spline-Kurven und -Flächen}
\begin{center}
 20.8
\end{center}
z. B. Kreis
\begin{center}
\begin{tabular}{ccc}
implizite Darstellung & explizite Darstellung & Parameterdarstellung \\
$x^2 + y^2 = 1$		& $y = \sqrt{1 - x^2}$	& $y = \cos \alpha$  \\
			& $y = \sqrt{1 - x^2}$	& $x = \sin \alpha$ \\
		 \hline
			& \emph{Spezialfall}	& \emph{Alternative (rationale Darstellung)} \\
			& $x = t$		& $x = \dfrac{2t}{1+t^2} \qquad t = \tan \frac{\alpha}{2}$\\
			& $y = \sqrt{1 - t^2}$	& $y = \dfrac{1-t^2}{1+t^2} \qquad -\infty < t < \infty$
\end{tabular}
\end{center}
\begin{center}
 20.9
\end{center}
\begin{itemize}
 \item Paramterdarstellung ist gut für das Nachfahren der Kurve; Darstellung als Folge von Punkten
 \item implizite Darstellung ist gut für Raytracing
\end{itemize}
\Defi	\textbf{Spline-Kurven} sund parametrische Kurven, wo siw Parameterfunktionen Polynome sind (es gibt auch
	rationale Splines). Die Kurven sind durch Leitpunkte (Kontrollpunkte) festgelegt. Dadurch sind sie einigermaßen
	intuitiv manipulierbar.

\begin{enumerate}
\item \textbf{Bézier-Splines}
	\begin{center}
	20.10
	\end{center}
	$k+1$ Kontrollpunkte definieren einen Bézier-Spline der Ordnung $k$ (Polynome vom Grad $\le k$)
\item \textbf{B-Splines}
	\begin{center}
	 20.11
	\end{center}
	Ordnung $k$: stückweise Polynome der Ordnung $k$, Kontrollpunkte $< \infty$
\item \textbf{Hermite-Splines}
	\begin{center}
	 20.12
	\end{center}
	interpolieren zwischen 2 Endpunkten mit vorgegebenen Tangetenrichtungen und Geschwindigkeiten\\
	$...$ kubische Splines
	\begin{center}
	 20.13
	\end{center}

\end{enumerate}