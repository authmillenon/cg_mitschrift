\chapter{Texturen}
\emph{(Abbildungen folgen später)}\\
Aufbringen eines Musters auf eine Fläche.\\[1em]
15.1
Das Muster -- die Textur -- ist als zweidimensionales \emph{Bild} gegeben (Pixelarray), und eine Abbildung
von diesem Bild auf eine Fläche im Raum.\\[1em]
15.2
Die Bildfläche muss nicht unbedingt flach sein:
\begin{enumerate}
 \item gekrümmt 15.3
 \item Dreiecksgitter 15.4
	Dann muss man das Bild in ein Dreicksgitter mit derselben kombinatorischen Struktur zerlegen.
	Die Abbildung erfolgt dann durch lineare Interpolation auf jedem Dreieck.
\end{enumerate}
Bei beiden Lösungen existiert das Problem der Verzerrung.\\[1em]
Einfaches Modell für Architektur: Fassade wird auf die Hasuwand aufgeklebt.
15.5
\paragraph*{Problem} Schatten sind möglicherweise nicht richtig.

Bei der Darstellung benötigt man die Umkehrabbildung von der Fläche in Weltkoordinaten
(bzw. von der projezierten Fläche in NDC) auf die Textur.

Bei der linearen Interpolation von \emph{Dreiecken} verwendet man gerne \emph{baryzenztische Koordinaten}
\begin{center}
 15.6
\end{center}
\[(\lambda_1, \lambda_2, \lambda_3) \text{ mit } \lambda_1 + \lambda_2 + \lambda_3 = 1\]
stellt den Punkt $\lambda_1 A_1 + \lambda_2 A_2 + \lambda_3 A_3$ dar.
\[\lambda_1, \lambda_2, \lambda_3 \ge 0 \Leftrightarrow \text{ Punkt im Dreieck}\]
\begin{center}
 15.7
\end{center}
Die Ecken des Dreiecks auf dem Objekt haben barizentrische Koordinaten
\[(\lambda_1, \lambda_2, \lambda_3) = (1,0,0), (0,1,0), (0,0,1)\]
dazwischen wird linear \emph{interpoliert} (in Weltkoordinaten). Dann nimmt man in der Textur den Punkt an der
Stelle $\lambda_1 B_1 + \lambda_2 B_2 + \lambda_3 B_3$.
\begin{center}
 15.8+9
\end{center}
Möglcihe Probleme dadurch, das Auflösung des erzeugten bilder nicht mit der Auflösung der Textur übereinstimmt.
\begin{enumerate}[a)]
 \item Auflösung der Textur zu klein\\
	$\Rightarrow$ Bild wirkt grob gepixelt (mögliche Lösung: Interpolation)
 \item Auflösung der Textur ist größer\\
	$\Rightarrow$ Es kann zu Aliasing-Effekten kommen. (mögliche Auswege s. Abschnitt \ref{sec:antialiasing})
\end{enumerate}

\chapter{Schatten}
\begin{center}
 13.10
\end{center}
Für die Berecnhung von chatten kann man die gleichen Methoden verwenden, wie zur Bestimmung sichtbarer Flächen:
\begin{itemize}
 \item von der Lichtquelle aus "`sichtbar"' $\Rightarrow$ beleuchtet.
 \item von der Lichtquelle aus "`verdeckt"' $\Rightarrow$ Schatten.
\end{itemize}
\begin{enumerate}
 \item Berechne (für jede Lichtquelle unabhängig) in einem getrennten Puffer das "`Bild"', dass von der Lichtquelle aus
	"`gesehen"' wird.\\
	z. B. mit dem Tiefenpuffer \lstinline[mathescape=true]!PutPixel($x$,$y$,$z$,$\rnode{I}{I}$)!
	\begin{center}
	 \rnode{Index}{Index} (Nummer) der Fläche, die gerade dargestellt wird.
	 \ncangle[angleA=90,angleB=-90]{->}{Index}{I}\\
	 13.11
	\end{center}
 \item Erzeuge das Bild, dass vom Auge aus sichtbar ist Pixel für Pixel.
	\begin{center}
	 13.12
	\end{center}
	$P$ sei der Punkt (in Weltkoordinaten), der gerade dargestellt wird. Er gehört zum Objekt mit Nummer $n$.
	Bestimme das Bild von $P$ im Bildpuffer, der zur Lichtquelle gehört und vergleiche den Index, der dort
	gespeichert ist.

	Wenn $= n \Rightarrow $ Punkt $P$ ist von dieser Lichtquelle beleuchtet.
\end{enumerate}
\chapter{Clipping}
Eine Strecke auf ein Rechteck zurechtstutzen.
\begin{center}
 15.13
\end{center}
\section{Algorithmus von Cohen-Sutherland}
$\mathbb{R}^2 \to \{0,1\}^4 \qquad$ (4 Bits)
\begin{align*}
 x &< x_\mathrm{min} \\
 x &> x_\mathrm{max} \\
 y &< y_\mathrm{min} \\
 y &< y_\mathrm{max} \\
\end{align*}
\begin{enumerate}
 \item Wenn $A_1$ und $A_2$ ein gemeinsames 1-Bit haben.\\
	$\Rightarrow$ Strecke $A_1A_2$ schneidet Rechteck nicht.
 \item Wenn beide Codes 0000 sind.\\
	$\Rightarrow$ Strecke liegt zur Gänze im Inneren.
 \item Bestimme eine Stelle, wo sich beide Codes unterscheiden,
 \item Schneide die Gerade mit der entsprechenden horizontalen oder vertikalen Geraden.
 \item Ersetze den Endpunkt mit dem 1-Bit durch den Schnittpunkt.
 \item Wiederhole
\end{enumerate}




