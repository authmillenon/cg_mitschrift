\section{Globale Beleuchtung}
\begin{center}
 %19.1
\end{center}
\subsection{Rendering-Gleichung}
Die \textbf{Rendering-Gleichung} bildet die tatsächlichen Verhältnisse bei der Beleuchtung am genauesten ab.
\begin{center}
 %19.2
\end{center}
\paragraph*{Variablen}
\begin{itemize}
 \item $p, p', p'', p'''...$ Punkte auf den Oberflächen der Objekte
 \item $\overline{pp'}...$ Sehstrahlen.
 \item $f(p',p'')...$ Leuchtdichte auf dem Strahl $p'p''$
\end{itemize}
\paragraph*{Gegeben}
\begin{itemize}
 \item $\rho(p',p'',p''')...$ Welchen Anteil des Lichtes auf $p'p''$ wird in Richtung $p''p'''$ zurückgeworfen?
\end{itemize}
\paragraph{Beispiel} perfekter Spiegel:
	\[
		\rho(p',p'',p''') = \begin{cases}
		                     1,& \text{wenn die Winkehalbierende von $p'p''p'''$ senkrecht auf der Fläche steht}\\
		                     0,& \text{sonst}
		                    \end{cases}
	\]
\paragraph{Rendering-Gleichung}
\[\boxed{f(p'',p''') = \int\limits_{p'{:}\ \text{$p'p''$ ist sichtbar}} f(p',p'') \rho(p',p'',p''')
	\underbrace{\mathrm{d}p'}_{\text{Flächenintegral}}
	+ \underbrace{e(p'',p''')}_{
		\text{\begin{minipage}{2.5cm}
			Wass wird von $p''$ als Lichtquelle in Richtung $p'''$ ausgestrahlt?
		\end{minipage}}}}
\]
Integralgleichung. $f$ ist eine Funktion von 4 Variablen
\[f(p'',p''',\lambda) = \int ... \rho(p',p'',p''',\lambda) ..., \lambda \in \mathbb{R} \text{ bzw. } \lambda \in \{R,G,B\}\]
$\Rightarrow$ In wirklichkeit auch von $\lambda$ abhängig.
\subsection{Radiosity-Verfahren}
Vereinfachung der Rendering-Gleichung
\begin{itemize}
 \item Betrachtet alle Wechselwirkungen zwischen Flächen
 \item nur diffuse Flächen
 \item Die Flächen werden in kleine Flächenstücke zerlegt; auf die Art wird das Problem diskreditiert\\
	$\rightarrow$ großes lineares Gleichungssystem
\end{itemize}
\textbf{Formfaktor $\boldsymbol{f_{ij}}$:} Welcher Anteil des Lichtstroms, der von Fläche $A_i$ ausgesendet wird,
	kommt bei der Fläche $A_j$ an?
\paragraph*{Variablen}
\begin{itemize}
 \item $b_i...$ Beleuchtungsstärke der Fläche $A_i$.
	Wieviel Lichtstrom wird pro Flächeneinheit ausgeschickt?
 \item $\rho_i...$ Reflektionskoeffizient der Fläche $A_i$
 \item $e_i...$ Eigenstrahlungslichtstärke von $A_i$, wenn $A_i$ eine Lichtquelle ist.
\end{itemize}
Summe des eintreffenden Lichtes auf $A_i$:
	\[\sum\limits_j b_j \cdot |A_j| \cdot f_{ji}\]
Ausgeschicktes Licht auf $A_i$:
	\[\boxed{b_i \cdot |A_i| = \rho_i \cdot \sum\limits_j b_j \cdot |A_j| \cdot f_{ji} + e_i \cdot |A_i|}\]
Dies muss man für $R, G, B$ bzw. für jede Wellenlänge $\lambda$ seperat lösen.

\subsubsection{Formfaktoren}
Hängen von der Geometrie ab.
\begin{center}
 %14.3, noch nicht fertig
 \psset{Alpha=170,Beta=20}
 \begin{pspicture}(-2,0)(2,3)
  \pstThreeDSquare(-1,-1,0)(2,0,0)(0,2,0)
  \pstThreeDSquare(2,2,1)(2,-2,0)(0,0,2)
  \pstThreeDLine{*-*}(0,0,0)(3,1,2)
  \pstThreeDLine[linestyle=dashed](0,0,0)(0,0,2)
  \pstThreeDLine[linestyle=dashed](3,1,2)(1,0,2)
 \end{pspicture}
\end{center}
$p \in A_i, q \in A_j$\\
$\alpha, \beta$ Winkel zu den Flächennormalen\\
$r = \| p - q \|$
\paragraph*{Annahme} $A_i$ und $A_j$ relativ klein, Sicht ist nicht unzterbrochen $... r, \alpha, \beta$ schwanken nur wenig.
\paragraph*{Näherungsformel} (Fehler gering wenn $p$ und $q$ weit aus einander)
\[f_{ij} \approx \cos \alpha \cdot \cos \beta \cdot \frac{1}{r^2} \cdot |A_j| \cdot \frac{1}{\pi} \]
\paragraph*{Exakte Formel}
\begin{align*}
f_{ij} &= \int\limits_{p \in A_i} \int\limits_{q \in A_j} \frac{\cos \alpha (p,q) \cdot \cos \beta (p,q) \cdot s(p,q)}{\|p - q\|^2} \mathrm{d}q\ \mathrm{d}p
	\cdot \frac{1}{|A_i|} \cdot \frac{1}{\pi}\\
s(p,q) &= \begin{cases}
		1, & \text{$p$ und $q$ sehen sich}\\
		0, &  \text{sonst}
          \end{cases}
\end{align*}
\[\boxed{f_{ij} \cdot |A_i| = f_{ji} \cdot |A_j|}\]
\begin{align*}
b_i \cdot |A_i| &= \rho_i \cdot \sum\limits_j b_j \cdot |A_j| \cdot f_{ji} + e_i \cdot |A_i|\\
b_i \cdot \cancel{|A_i|} &= \rho_i \cdot \sum\limits_j b_j \cdot \cancel{|A_i|} \cdot f_{ij} + e_i \cdot \cancel{|A_i|}\\
	&\Rightarrow \boxed{b_i = \rho_i \sum\limits_j b_j \cdot f_{ij} + e_i} \qquad \text{Radiosity-Gleichungssystem}
\end{align*}
lineares Gleichungssysmte in $n$ Variablen $b_1, ..., b_n$ ($n =$ \#Flächenstücke), $O(n^2)$ Koeffizienten $f_{ij} \qquad 0 \le \rho_i \le 1$

\subsubsection{Geometrische "`Berechnung"' von $\boldsymbol{f_{ij}}$}
$A_i$ ist ein kleines Flächenstück
\begin{center}
 %19.4
\end{center}
Projektion auf die obere Einheitshalbkugel um $A_i$\\
$\hat =$ Multiplikation mit $\cos \beta \cdot \frac{1}{r^2}$\\
$\hat =$ scheinbare Größe von $A_j$ aus Sicht von $A_i$\\
Anschließende Projektion senkrecht auf die Ebene durch $A_i\ \hat =$ Multiplikation mit $\cos \alpha$\\
gesamter Fläche des Kreises $=\ \pi$
\[\sum\limits_i f_{ij} \stackrel{\mathbf{!}}= 1\]
Wahl des Konstanten Faktors $\frac{1}{\pi}$ wird durch die physikalische Forderung $\sum\limits_{j} = 1$ erzwungen.

Das Gleichungssystem löst man am besten iterativ:
\begin{itemize}
 \item Beginne mit einer beliebigen Ausgangslösung $\vec b^{(0)}$ z. B. durch Phong-Beleuchtung berechnet. oder
	\[\vec b^{(0)} = \vthree{b_1^{(0)}}{\vdots}{b_n^{(0)}} \cdot \vthree{0}{\vdots}{0}\]
 \item Setze $b^{(k)}$ rechts ein und erhalte die nöchste Näherungslösung
	\[\boxed{b_i^{(k+1)} = \sum\limits_{i}{j=1} b_j^{(k)} \cdot f_{ij} + e_i}, \qquad i = 1, ..., n\]
\end{itemize}
Das konvergiert umso schneller, je kleiner die $\rho_i$-Werte sind.

