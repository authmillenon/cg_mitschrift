\chapter{Koordinatensysteme, geometrische Transformationen}
\section{kartesische Koordinaten}
\begin{center}
\begin{pspicture}[unit=0.5cm](0,0)(5,3)
	\psaxes[Dx=2,Dy=2](0,0)(0,0)(10,6)[$x$,0][$y$,0]
	\psdot(5,3)\rput[bl](5,3){$p = \vtwo{x}{y} = \vtwo{5}{3}$}
\end{pspicture}
\end{center}

\section{Geometrische Transformationen}
\begin{itemize}
 \item \emph{Tranlation}: $p \mapsto p + t \qquad t \in \mathbb{R}^2$,  Translationsvektor
	\begin{center}
	 \begin{pspicture}[unit=0.5cm](0,0)(4,4)
	  \psaxes[labels=none,ticks=none](0,0)(0,0)(8,8)
	  \pspolygon[linestyle=dotted](6,5)(7,7.5)(2,8)(3,3)
	  \rput[lb](5,3){\psline[linecolor=red](0,0)(1,2)\uput{3pt}[r]{0}(0.5,1){\color{red}$t$}}
	  \rput[lb](6,5.5){\psline[linecolor=red](0,0)(1,2)\uput{3pt}[r]{0}(0.5,1){\color{red}$t$}}
	  \rput[lb](1,6){\psline[linecolor=red](0,0)(1,2)\uput{3pt}[r]{0}(0.5,1){\color{red}$t$}}
	  \rput[lb](2,1){\psline[linecolor=red](0,0)(1,2)\uput{3pt}[r]{0}(0.5,1){\color{red}$t$}}
	  \pspolygon(5,3)(6,5.5)(1,6)(2,1)
	  \psdot(5,3)
	  \psdot(6,5.5)
	  \psdot(1,6)
	  \psdot(2,1)
	 \end{pspicture}
	\end{center}

 \item \emph{Rotation} (um den Ursprung $\vtwo{0}{0}$):
	\[p \mapsto M \cdot p \qquad M = \begin{pmatrix}
	                                  \cos{\alpha} & -\sin{\alpha} \\
	                                  \sin{\alpha} & \cos{\alpha}
	                                 \end{pmatrix}\text{, Rotationsmatrix}\]
	\begin{center}
	 \begin{pspicture}[unit=0.5cm](-2,-2)(2,2)
	  \psaxes[labels=none,ticks=none](0,0)(-4,-4)(4,4)
	  \SpecialCoor
	  \rput[tl](-4,4){\color{red}$\alpha = 20^\circ$}
	  \psarc[linecolor=red](0,0){3}{30}{50}\uput{3pt}[40]{0}(3;40){\color{red}$\alpha$}
	  \psarc[linecolor=red](0,0){4}{135}{155}\uput{3pt}[145]{0}(4;145){\color{red}$\alpha$}
	  \psarc[linecolor=red](0,0){2}{-70}{-50}\uput{3pt}[-60]{0}(2;-60){\color{red}$\alpha$}
	  \pspolygon(3;30)(4;135)(2;-70)
	  \pspolygon[linestyle=dotted](3;50)(4;155)(2;-50)
	 \end{pspicture}
	\end{center}
 \item \emph{Rotation} um den Punkt $c$: $p \mapsto M(p -c) + c = Mp + (c - Mc), \qquad c \mapsto c$
 \item \emph{gleichförmige Skalierung}:
	\begin{align*}
	&&p &\mapsto \lambda \cdot p = \begin{pmatrix}
	                               \lambda & 0 \\
	                               0 & \lambda
	                              \end{pmatrix} \cdot p, \qquad \lambda \neq 0\\
	&\lambda = 1& p &\mapsto -p = \begin{pmatrix}
	                                     -1 & 0 \\
	                                     0 & -1
	                                    \end{pmatrix} \cdot p = \text{Spiegelung am Ursprung $=$ Rotation um $180^\circ$}
	\end{align*}
 \item \emph{Ungleichförmige Skalierung}:
	\[M = \begin{pmatrix}
	       \lambda_1 & 0 \\
	       0 & \lambda_2
	      \end{pmatrix} \qquad p \mapsto M \cdot p\]
	\[\vtwo{x}{y} = \vtwo{\lambda_1 x}{\lambda_2 y}\]
	\[M = \begin{pmatrix}
	       -1 & 0 \\
	       0 & 1
	      \end{pmatrix} \text{resultiert in der Spiegelung an der $x$-Achse}\]
	\[M = \begin{pmatrix}
	       1 & 0 \\
	       0 & -1
	      \end{pmatrix} \text{resultiert in der Spiegelung an der $y$-Achse}\]
 \item \emph{Scherung}
	\[M = \overset{\text{Scherung auf der $x$-Achse}}{\begin{pmatrix}
		1 &  \lambda \\
		0 & 1
	      \end{pmatrix}} \left(\text{oder } \overset{\text{Scherung auf der $y$-Achse}}{\begin{pmatrix}
		1 & 0 \\
		\lambda & 1 \\
	      \end{pmatrix}}
	      \right)\]
	      \[\vtwo{x}{y} \mapsto \begin{pmatrix}
		1 &  \lambda \\
		0 & 1
	      \end{pmatrix} \vtwo{x}{y} = \vtwo{x+ \lambda y}{y}\]
	 \begin{center}
	 \begin{pspicture}[unit=0.5cm](0,0)(4,2.1)
	  \rput[tl](0.1,4){\color{red}$\lambda = 0{,}5$}
	  \psaxes[labels=none,ticks=none](0,0)(0,0)(8,4.2)
	  \psline[linecolor=red](4,1)(4.5,1)
	  \psline[linecolor=red](4,2)(5,2)
	  \psline[linecolor=red](4,3)(5.5,3)
	  \psline[linecolor=red](4,4)(6,4)
	  \psdot(4,1)\psdot[dotstyle=o](4.5,1)
	  \psdot(4,2)\psdot[dotstyle=o](5,2)
	  \psdot(4,3)\psdot[dotstyle=o](5.5,3)
	  \psdot(4,4)\psdot[dotstyle=o](6,4)
	 \end{pspicture}
	\end{center}
\end{itemize}
Flächeninhalt:
\begin{itemize}
 \item Translationen, Rotationen, Scherungen und Spiegelungen ändern den Flächeninhalt \underline{nicht}.
 \item Skalierung ändert den Flächeninhalt um den Faktor $\lambda_1 \cdot \lambda_2$
\end{itemize}

\Defi Eine Verknüpfung mehrerer dieser Transformationen bildet eine \textbf{affine Transformation}.
	Allgemein ist diese:
	\[ p \mapsto M \cdot p = b, \qquad M \in \mathbb{R}^{2 \times 2}, b \in \mathbb{R}^2, \det M \neq 0 \]
	Der Flächeninhalt ändert sich um den Faktor $\det M$

\Defi Die Verknüpfung von Translation, Rotation und Spiegelung heißt \textbf{starre Bewegung} oder \textbf{Isometrie}.
	Allgemein ist diese:
	\[ p \mapsto Mp + t \text{ mit \textbf{orthogonaler Matrix} $M$ (d. h. $\det M = \pm 1$)}\]
	die Isometrien zerfallen:
	\begin{itemize}
	 \item \textbf{orientierungserhaltende} ($\det M = 1$) und
	 \item \textbf{orientierungsumkehrende} ($\det M = -1$) Isometrien
	\end{itemize}

\section{Homogene Koordinaten}
\Defi \textbf{Homogene Koordinaten}: Statt $p = \vtwo{x}{y}$ verwendet man eine dritte Koordinate
	$p = \vthree{x}{y}{1}$

\paragraph*{Konvention} Die Koordinaten $\vthree{x}{y}{z}$ und
	$\vthree{\lambda x}{\lambda y}{\lambda z}$ stellen denselben Punkt dar ($\lambda \neq 0$)

Der Punkt $\vthree{x}{y}{z}$ mit $z \neq 0$ hat die kartesischen Koordinaten $\vtwo{\frac{x}{z}}{\frac{y}{z}}$

\subsection{Allgemeine affine Transformation in homogenen Koordinaten}
\[\vthree{x}{y}{z} \mapsto \underbrace{\begin{pmatrix}
				m_{11} & m_{12} & b_1\\
				m_{21} & m_{22} & b_2\\
				0 & 0 & 1
                           \end{pmatrix}}_{M'} \cdot \vthree{x}{y}{z}\]
\[\vthree{x}{y}{1} \mapsto \vthree{m_{11}x + m_{12}y + b_1}{m_{21}x + m_{22}y + b_2}{1}\]

Die Matrizen $M'$ und $\lambda M'$ beschreiben dieselbe Transformation $(\lambda \neq 0)$

\[p \mapsto M' p \text{ mit } M' = \begin{pmatrix}
                                    m_{11} & m_{12} & m_{13} \\
                                    m_{21} & m_{22} & m_{23} \\
                                    0 & 0 & m_{33}
                                   \end{pmatrix} \text{ und $\det M' \neq 0$}
\]
\[\det M' \neq 0 \Leftrightarrow m_{33} \neq 0 \land \begin{vmatrix}
                                                    m_{11} & m_{12} \\
                                                    m_{21} & m_{22}
                                                   \end{vmatrix} \neq 0
\]
$\Rightarrow$ o. B. d. A. kann man auch $m_{33} = 1$ annehmen (Dann kann man die dritte Zeile auch weglassen).

\section{Die projektive Ebene}
\Defi Die (reelle) \textbf{projektive Ebene} $P^2$ besteht aus den Äquivalenzklassen vo Punkten
	$\vthree{x}{y}{z} \neq \vthree{0}{0}{0}$, wobei $\vthree{x}{y}{z}$ und $\vthree{\lambda x}{\lambda y}{\lambda z}$
	denselben Punkt darstellen ($\lambda \neq 0$)

\subsection{Geraden in der projektiven Ebene}
Gerade in $\mathbb{R}^2$ (karthesische Koordinaten):
 \[y = ax + b \text{(Gerade darf nicht senkrecht sein)}\]
 \[ax + bx = -c\]

\[\Updownarrow\]

Gerade in Homogenen Koordinaten
\[\vtwo{x}{y} \longrightarrow \vthree{x}{y}{1}\]
\[ax + by + c = 0 \Leftrightarrow \vthree{a}{b}{c} \cdot \vthree{x}{y}{1} = 0\]

Allgemeine Gleichung einer Geraden in $P^2$
\[\vthree{a}{b}{c} \cdot \vthree{x}{y}{z} = 0 \Leftrightarrow ax + by + cz = 0 \qquad \vthree{a}{b}{c} \neq \vthree{0}{0}{0}\]

Wenn $\vthree{x}{y}{z}$ die Gleichung erfüllt, dann erfüllt auch $\vthree{\lambda x}{\lambda y}{\lambda z}$ die Gleichung.

$\vthree{a}{b}{c}$ und $\vthree{\lambda a}{\lambda b}{\lambda c}$ stellen dieselbe Gerade dar.

\begin{description}
 \item[projektive Punkte] $\vthree{x}{y}{z} \neq \vthree{0}{0}{0}$ Skalierung egal.
 \item[projektive Gerade] $\vthree{a}{b}{c} \neq \vthree{0}{0}{0}$ Skalierung egal
\end{description}
\Satz Punkt $\vthree{x}{y}{z}$ liegt auf der Geraden $\vthree{a}{b}{c}$:
\[\vthree{a}{b}{c} \cdot \vthree{x}{y}{z} = 0\]

\Satz	Zwei verschiedene Geraden schneiden sich in genau einem Punkt.
\Bew	Gerade $\forall \lambda: \vthree{a_1}{b_1}{c_1}, \vthree{a_2}{b_2}{c_2} \neq \lambda \vthree{a_1}{b_1}{c_1}$.

Schnittpunkt:
\begin{align*}
 a_1 x + b_1 y + c_1 z &= 0\\
 a_2 x + b_2 y + c_2 z &= 0
\end{align*}

Koeffizientenmatrix $A = \begin{pmatrix}
                          a_1 & b_1 & c_1 \\
                          a_2 & b_2 & c_2
                         \end{pmatrix}, \rg A = 2$\\
$\Rightarrow$ Lösungsmenge ist eindimensional
\[ L = \left\{\left.\lambda \vthree{x_0}{y_0}{z_0} \right| y \in \mathbb{R}\right\} \text{ist ein projektiver Punkt}\]

$\vthree{x_0}{y_0}{z_0}$ kann als $\vthree{a_1}{b_1}{c_1} \times \vthree{a_2}{b_2}{c_2} = \vthree{
		\begin{vmatrix}
		 b_1 & b_2 \\
		 c_1 & c_2
		\end{vmatrix}}{
		\begin{vmatrix}
		 c_1 & c_2 \\
		 a_1 & a_2
		\end{vmatrix}}{
		\begin{vmatrix}
		 a_1 & a_2 \\
		 b_1 & b_2
		\end{vmatrix}}
$ berechnet werden (Kreuzprodukt)

\begin{center}
	\begin{pspicture}(-1,0)(3,3.5)
		\pstThreeDLine[fillstyle=hlines,hatchcolor=gray,linecolor=gray]{-}(0,0,0)(-2,1,0.5)(-4,0,1.5)(-2,-1,1)
		\pstThreeDSquare(0,0,0)(-0.4,0.2,0.1)(0.363,0.242,0.968)
		\pstThreeDDot[dotsize=1pt](-0.0185,0.221,0.534)
		\pstThreeDSquare(0,0,0)(-0.4,-0.2,0.2)(0.363,0.242,0.968)
		\pstThreeDDot[dotsize=1pt](-0.0185,0.021,0.584)
		\pstThreeDDot(-0.0185,0.221,0.534)
		\pstThreeDLine(0,0,0)(-2.2,1.1,0.55)\pstThreeDPut(-2.4,1.2,0.6){$\vec u$}
		\pstThreeDLine(0,0,0)(-2.2,-1.1,1.1)\pstThreeDPut(-2.4,-1.2,1.2){$\vec v$}
		\pstThreeDLine(1.815,1.21,4.84)\pstThreeDPut(2,1.41,5.2){$\vec u \times \vec v$}y
	\end{pspicture}
\end{center}


\Satz Durch zwei verschiedene Punkte gint es genau eine Geraden

\Bew gleich wie oben: $\vthree{a}{b}{c}$ mit $\vthree{x}{y}{z}$ vertauschen.

\paragraph*{Dualitätsprinzip} Man kann in einem Satz der projektiben Geometrie der Ebene "`Punkte"' und "`Geraden"'
	vertauschen und es bleibt ein gültiger Satz.
	
\pagebreak
\subsection{Modelle der projektiven Ebene}
\begin{enumerate}
 \item \emph{Räumliches Modell der projektiven Ebene}
	$\left\{\left.\lambda \vthree{x_0}{y_0}{z_0}\right| \lambda \in \mathbb{R}\right\}$...
		Geraden durch den Ursprung im $\mathbb{R}^3$ entsprechen den projektiven Punkten.
	\begin{center}
	 \begin{pspicture}(-5,-2)(5,4)
	  \pstThreeDCoor[xMin=-4,xMax=4,yMin=-4,yMax=4]
	  \pstThreeDLine{-}(-4,2,0)(4,-2,0)
	  \pstThreeDLine{-}(-2,-4,0)(2,4,0)
	  \pstThreeDLine{-}(3,-1.5,0)(-1.5,-3,0)(-3,1.5,0)(1.5,3,0)(3,-1.5,0)
	 \end{pspicture}
	\end{center}
	\begin{center}
	 projektive Gerade $\equiv$ Ebene durch den Ursprung
	\end{center}

\item \emph{Kugelmodell der projektiven Ebene}
	entsteht durch Schnitt des räumlichen Modells mit der Einheitskugel $S^2 =
		\left\{\left.\vthree{x}{y}{z}\right| x^2 + y^2 + z^2 = 1\right\}$
	
	\begin{center}
	 \begin{pspicture}(-5,-3)(5,3)
	  \psset{arrows=-}
	  \psellipticarc[linestyle=dashed]{-}(0,0)(2,1){0}{180}
	  \pstThreeDCoor[xMin=-4,yMin=-4,zMin=-3]
	  \pstThreeDLine{-}(-4,2,0)(4,-2,0)
	  \pstThreeDLine{-}(-2,-4,0)(2,4,0)
	  
	  \pstThreeDLine{-}(4,2,0)(-4,-2,0)
	  \pstThreeDDot(1.8,0.9,0)
	  \pstThreeDDot(-1.8,-0.9,0)

	  \pstThreeDCircle[Alpha=55,Beta=10](1.153,1.153,-1.153)(-.2,.2,0)(.231,.231,.231)
	  
	  \pscircle(0,0){2}
	  \psellipticarc{-}(0,0)(2,1){180}{0}

	  \pstThreeDNode(0,-2,0){GK}
	  \rput[tl](-3,2){\rnode{GK_label}{Großkreis}}
	  \ncline{->}{GK_label}{GK}
	  \pnode(1.4,1.4){EK}
	  \rput[tl](3,3){\rnode{EK_label}{Einheitskugel}}
	  \ncline{->}{EK_label}{EK}

	  \pstThreeDNode[Alpha=55,Beta=10](1.153,1.153,-1.153){kGK}
	  \rput[tl](3,-2){\Rnode[href=-0.9]{kGK_label}{kein Großkreis}}
	  \ncline[nodesepB=5px]{->}{kGK_label}{kGK}
	 \end{pspicture}
	\end{center}
	\begin{align*}
	 \text{projektiver Punkt} &\equiv \text{Paar gegenüberliegender Punkte auf der Einheitskugel}\\
	 \text{projektive Gerade} &\equiv \text{Großkreise}
	\end{align*}
\end{enumerate}


\subsection{Projektive Punkte zu karthesische Koordinaten}
\begin{center}
	 \begin{pspicture}(-5,-2)(5,4)
	  \psset{arrows=-}
	  \psellipticarc[linestyle=dashed]{-}(0,0)(2,1){0}{180}
	  \pstThreeDCoor[xMin=-4,yMin=-4,zMin=-3]

	  \pstThreeDSquare(-2,-2,2)(6,0,0)(0,6,0)

	  \pstThreeDLine(0.2,1,-2)(-0.4,-2,4)
	  \pstThreeDLine(-3,-9,-6)(3,9,6)
	  
	  \pstThreeDDot(0,0,2)
	  \pstThreeDDot[drawCoor=true](-0.2,-1,2)
	  \pstThreeDDot[drawCoor=true](1,3,2)

	  \pstThreeDPut(-3,2,2){$\vthree{\frac{x}{z}}{\frac{y}{z}}{1}$}
	  
	  \pscircle(0,0){2}
	  \psellipticarc{-}(0,0)(2,1){180}{0}

	  \pstThreeDNode[Alpha=75](1.4,1.4,0){aq}
	  \rput[tl](3,-2){\Rnode[href=-0.9]{aq_label}{Äquator = Ferngerade}}
	  \ncline[nodesepB=5px]{->}{aq_label}{aq}
	 \end{pspicture}
\end{center}
Schnitt der Geraden $\vthree{x}{y}{z} \cdot \lambda$ im $\mathbb{R}^3$ mit Ebene $z = 1$:
$z \cdot \lambda = 1 \Rightarrow \lambda = \dfrac{1}{z}$\\
$\rightarrow \vthree{x \cdot \frac{1}{z}}{y \cdot \frac{1}{z}}{1}$

\Satz Die Punkte $\vthree{x}{y}{z}$ mit $z = 0$ haben \emph{keine} Entsprechung in der euklidischen Ebene:
Jede projektive Gerade hat als Bild in der euklidischen Ebene eine Gerade, mit einer Ausnahme: die Gerade
$\vthree{0}{0}{1}$

\Defi Die Punkte des projektiven Raumes, die keine euklidische Entsprechung haben, heißen \textbf{Fernpunkte}.
	Die Gerade $\vthree{0}{0}{1}$ \textbf{Ferngerade}.

\Satz Zwei Geraden der euklidischen Ebene sind genau dann \emph{parallel}, wenn ihr Schnittpunkt ein Fernpunk ist.

\Satz Die Punkte, die auf der Ferngeraden liegen, sind genau die Fernpunkte

\Satz Es gibt zu jeder Schaar paralleler Geraden genau einen Fernpunkt.
\begin{center}
 \begin{pspicture}(0,0)(5,4)
	\psset{arrows=-}
	\rput[lb](0.5,1.75){\psline(0,0)(2.25,2.25)}
	\rput[lb](1,1.5){\psline(0,0)(2.25,2.25)}
	\rput[lb](1.5,1.25){\psline(0,0)(2.25,2.25)}
	\rput[lb](2,1){\psline(0,0)(2.25,2.25)}
	\rput[lb](2.5,0.75){\psline(0,0)(2.25,2.25)}
	\psline(0.25,1.75)(1.15,0.9)(2.75,0.5)
	\psdot(1,0.75)
 \end{pspicture}
\end{center}

Anschaulich ist ein Fernpunkt äquivalent zu perspektivischen Sammelpunkten:
\begin{center}
 \begin{pspicture}(-3,-2)(0,0)
	\psset{arrows=-}
	\psdot(0,0)
	\rput[bl](0.1,0){Fernpunkt}
	\psline[linestyle=dotted](0,0)(-2,-2)
	\psline[linestyle=dotted](0,0)(-3,-2)
	\pspolygon(-2,-2)(-0.5,-0.5)(-0.75,-0.5)(-3,-2)
 \end{pspicture}
\end{center}

\subsection{Allgemeine projektive Transformationen}
$\xyz \mapsto M \cdot \xyz$ mit $M = \mathbb{R}^{3 \times 3}, \det M \neq 0$\\
(Punkte bleiben Punkte, Geraden bleiben Geraden, Inzidenz bleibt erhalten)

\Defi Affine Transformationen sind jene Transformationen, bei denen die Fernpunkte Fernpunkte bleiben.
\[ \vthree{x}{y}{z} \mapsto M \vthree{x}{y}{0} = \vthree{x'}{y'}{z'} \stackrel{!}= \vthree{x'}{y'}{0}\]
\[M = \begin{pmatrix}
       m_{11} & m_{12} & m_{13} \\
       m_{21} & m_{22} & m_{23} \\
       \cancel{m_{31}}{0} & \cancel{m_{32}}{0} & m_{33}
      \end{pmatrix}
\]
\begin{align*}
 \forall x, y: m_{31} x + m_{32} y + m_{33} \cdot 0 = 0 &\Rightarrow m_{31} = m_{32} = 0
\end{align*}

\begin{align*}
 \det M &\neq 0\\
 \det M &= \underbrace{m_{33}}_{\neq 0} \cdot \underbrace{\begin{vmatrix}m_{11}&m_{12}\\m_{21}&m_{22}\end{vmatrix}}_{0}
	&\Rightarrow \text{o. B. d. A. $m_33 = 1$}
\end{align*}

\[\vtwo{x}{y} \mapsto \overbrace{\begin{pmatrix}m_{11}&m_{12}\\m_{21}&m_{22}\end{pmatrix}
	\vtwo{x}{y}}^{\text{lineare Transformation}} + \overbrace{\vtwo{m_{13}}{m_{23}}}^{\text{+ Translation}}\]

Affine Transformation:
\begin{itemize}
 \item parallele Geraden bleiben parallel
	\begin{center}
	 \begin{pspicture}(0,-0.5)(12,3.5)
	  \psframe[fillstyle=hlines,linestyle=none](0,1)(1,2)
	  \psframe[fillstyle=hlines,linestyle=none](2,0)(3,1)
	  \psellipse[fillstyle=hlines](4,0.5)(0.5,0.5)
	  \psline{-}(0.5,1.8)(0.5,2.2)
	  \psline{-}(0,-0.2)(0,3.2)
	  \psline{-}(1,-0.2)(1,3.2)
	  \psline{-}(2,-0.2)(2,3.2)
	  \psline{-}(3,-0.2)(3,3.2)
	  \psline{-}(-0.2,0)(3.2,0)
	  \psline{-}(-0.2,1)(3.2,1)
	  \psline{-}(-0.2,2)(3.2,2)
	  \psline{-}(-0.2,3)(3.2,3)

	  \psline[linewidth=2px]{-}(1,3)(2,3)
	  \psline[linewidth=2px]{-}(0,0)(1,0)
	  
	  \psline[linewidth=3px](5,1.5)(7,1.5)

	  \pspolygon[fillstyle=hlines,linestyle=none,hatchangle=71.57](7.5,0.25)(7.5,1.25)(8.5,1.75)(8.5,0.75)
	  \pspolygon[fillstyle=hlines,linestyle=none,hatchangle=71.57](9.5,0.25)(9.5,1.25)(10.5,1.75)(10.5,0.75)
	  \rput{26.57}(11.5,1.75){\psellipse[fillstyle=hlines](0,0)(0.559,0.5)}
	  \psline{-}(8,1.3)(8,1.7)
	  \psline{-}(7.5,-0.95)(7.5,2.45)
	  \psline{-}(8.5,-0.45)(8.5,2.95)
	  \psline{-}(9.5,0.05)(9.5,3.45)
	  \psline{-}(10.5,0.55)(10.5,3.95)
	  \psline{-}(7.3,-0.85)(10.7,0.85)
	  \psline{-}(7.3,0.15)(10.7,1.85)
	  \psline{-}(7.3,1.15)(10.7,2.85)
	  \psline{-}(7.3,2.15)(10.7,3.85)

	  \psline[linewidth=2px]{-}(8.5,2.75)(9.5,3.25)
	  \psline[linewidth=2px]{-}(7.5,-0.75)(8.5,-0.25)
	 \end{pspicture}
	\end{center}
\psset{arrows=-}
 \item erhalten das Teilverhältnis auf parallelen Geraden
	\begin{center}
	 \begin{pspicture}(0,-0.5)(6,0.3)
	 \psline(0,0)(6,0)
	 \psline(1,-0.2)(1,0.2)\rput[t](1,-0.2){$A$}
	 \psline(3.2,-0.2)(3.2,0.2)\rput[t](3.2,-0.2){$B$}
	 \psline(5.6,-0.2)(5.6,0.2)\rput[t](5.6,-0.2){$C$}
	 \end{pspicture}
	\end{center}
	\[\frac{\overrightarrow{AB}}{\overrightarrow{AC}}\]
\end{itemize}
Starre Bewegungen (Isometrien, euklidische Transformationen):

$M = \begin{pmatrix}m_{11}&m_{12}\\m_{21}&m_{22}\end{pmatrix}$ ist orthogonal $M^T = M^{-1}$ erhalten Längen, Winkel und
	Flächen

\psset{arrows=-}
\paragraph*{Doppelverhältnis}
\begin{center}
 \begin{pspicture}(0,-0.5)(12,3.5)
	\pspolygon(0,0)(4,0)(4,3)(0,3)
	\psline(0,1)(4,1)\psline(0,2)(4,2)
	\psline(1,0)(1,3)\psline(2,0)(2,3)\psline(3,0)(3,3)\psline(4,0)(4,3)

	\psline(-0.5,-0.5)(3.5,3.5)

	\psdot(4,0)\uput{3pt}[r]{0}(4,0){$A$}
	\psdot(4,2)\uput{3pt}[r]{0}(4,2){$B$}
	\psdot(4,1)\uput{3pt}[r]{0}(4,1){$C$}
	\psdot(4,3)\uput{3pt}[r]{0}(4,3){$D$}

	\rput[bl]{-50}(7.5,0){
	\pstilt{100}{
	\rput[bl]{50}(0,0){
	\pstilt{60}{
	\pspolygon(0,0)(4,0)(4,3)(0,3)
	\psline(0,1)(4,1)\psline(0,2)(4,2)
	\psline(1,0)(1,3)\psline(2,0)(2,3)\psline(3,0)(3,3)\psline(4,0)(4,3)

	\psline(-0.5,-0.5)(3.5,3.5)

	\psdot(4,0)\uput{3pt}[r]{0}(4,0){$A$}
	\psdot(4,2)\uput{3pt}[r]{0}(4,2){$B$}
	\psdot(4,1)\uput{3pt}[r]{0}(4,1){$C$}
	\psdot(4,3)\uput{3pt}[r]{0}(4,3){$D$}
	}}}}
 \end{pspicture}
\end{center}
\[\boxed{\dfrac{\dfrac{\overrightarrow{AC}}{\overrightarrow{BC}}}{\dfrac{\overrightarrow{AD}}{\overrightarrow{BD}}}}
	= \text{Doppelverhältnis}\]


\paragraph*{Bemerkung} projektive Transformationen erhalten das sogenannte Doppelverhältnis
\begin{center}
% 2.9 
\end{center}

\paragraph*{Ausblick} projektiver Raum; wird beschrieben durch homogene Koordinaten $\vfour{x}{y}{z}{u}
	\neq \vfour{0}{0}{0}{0}$. Karthesische Koordinaten $\vthree{x}{y}{z}$ entsprechen homogenen
	Koordinaten $\vfour{x}{y}{z}{1}$ oder $\vfour{\lambda x}{\lambda y}{\lambda z}{\lambda}$ ($\lambda \neq 0$, bel.).


