\begin{center}
	\definecolor{this}{rgb}{0.25,0.75,0.25}
 \psset{unit=4cm,xMin=0,xMax=1.5,yMin=0,yMax=1.5,zMin=0,zMax=1.5,Alpha=60,linecolor=gray,Beta=20,nameX=$r$,nameY=$g$,nameZ=$b$}
 \begin{pspicture}(-1,-0.5)(1,1.5)
  \pstThreeDCoor[linecolor=black]
  \pstThreeDPut(0.5,0.5,0.75){$\vthree{\frac{1}{2}}{\frac{1}{2}}{\frac{1}{2}}$}
  \pstThreeDPut(0.25,0.75,0.5){$\vthree{\frac{1}{4}}{\frac{3}{4}}{\frac{1}{4}}$}
  \pstThreeDBox(0,0,0)(1,0,0)(0,1,0)(0,0,1)
  \pstThreeDLine(0,0,0)(1,1,1)
  \pstThreeDLine[linecolor=red](1,0,0)(1,1,0)(0,1,0)(0,1,1)(0,0,1)(1,0,1)(1,0,0) % HSV-Effekt?
  \pstThreeDPut(1,0.5,0){$\vthree{1}{g}{0}$}
  \pstThreeDPut(0,0.5,1){$\vthree{0}{g}{1}$}
  \pstThreeDPut(0.5,1,0){$\vthree{r}{1}{0}$}
  \pstThreeDPut(0.5,0,1){$\vthree{r}{0}{1}$}
  \pstThreeDPut(1.1,0,0.5){$\vthree{1}{0}{b}$}
  \pstThreeDPut(0,1.1,0.5){$\vthree{0}{1}{b}$}
  \pstThreeDPut(1,1,1){W}
  \pstThreeDPut(0,0,0){K}
  \pstThreeDPut(1,0,0){R}
  \pstThreeDPut(1,1,0){Y}
  \pstThreeDPut(1,0,1){M}
  \pstThreeDPut(0,1,0){G}
  \pstThreeDPut(0,1,1){C}
  \pstThreeDPut(0,0,1){B}
  \pstThreeDDot(0.5,0.5,0.5) % mit Koord beschriften
  \pstThreeDLine(0.5,0.5,0.5)(0,1,0) % mit Koord beschriften
  \pstThreeDPut(0,1,-0.25){$\vthree{0}{1}{0}$}
  \pstThreeDDot[linecolor=this](0.25,0.75,0.25) % mit Koord beschriften
 \end{pspicture}
\end{center}
Diese "`stärksten Farben"' sind die Farben din mindestens eine Komponente 0 und mindestens eine Komponente 1 haben
(Farben ohne Grau/Weiß/Schwarz-Anteil).

\subsection{Farbsechseck bzw. Farbkreis}
\begin{center}
 \begin{pspicture}(0,0)(4,4)
  \rput(2,2){
	\SpecialCoor
	\psline[linecolor=gray](1.85;40)(0,0)(2;90)
	\pspolygon(2;90)(2;30)(2;-30)(2;-90)(2;-150)(2;150)
	\psarc[linecolor=blue]{<-}(0,0){1.2}{100}{80} % HSV-Effekt?
	\psdot[linecolor=red](2;90)\uput{0.4cm}[-90](2;90){$0^\circ$}
	\psdot[linecolor=yellow](2;30)\uput{0.4cm}[-150](2;30){$60^\circ$}
	\psdot[linecolor=green](2;-30)\uput{0.4cm}[150](2;-30){$120^\circ$}
	\psdot[linecolor=cyan](2;-90)\uput{0.4cm}[90](2;-90){$180^\circ$}
	\psdot[linecolor=blue](2;-150)\uput{0.4cm}[30](2;-150){$240^\circ$}
	\psdot[linecolor=magenta](2;150)\uput{0.4cm}[-30](2;150){$300^\circ$}
	\rput(1;65){$40^\circ$}
  }
 \end{pspicture}
 \hspace{2cm}
 \begin{pspicture}(0,0)(4,4)
  \rput(2,2){
	\SpecialCoor
	\psline[linecolor=red](2;40)(0,0)(2;90)
	\pscircle(0,0){2}	% HSV-Effekt
	\psdot[linecolor=red](2;90)\uput{0.2cm}[90](2;90){$0^\circ$}
	\psdot[linecolor=yellow](2;30)\uput{0.2cm}[30](2;30){$60^\circ$}
	\psdot[linecolor=green](2;-30)\uput{0.2cm}[-30](2;-30){$120^\circ$}
	\psdot[linecolor=cyan](2;-90)\uput{0.2cm}[-90](2;-90){$180^\circ$}
	\psdot[linecolor=blue](2;-150)\uput{0.2cm}[-150](2;-150){$240^\circ$}
	\psdot[linecolor=magenta](2;150)\uput{0.2cm}[150](2;150){$300^\circ$}
	\rput(1;65){$40^\circ$}
  }
 \end{pspicture}
\end{center}
\begin{itemize}
 \item Die Punkte auf diesem Sechseck werden häufig durch einen Winkel ($0^\circ-360^\circ$) parametrisiert.
 \item Startpunkt willkürlich (R = $0^\circ$, Y = $60^\circ$, ...)
 \item Dieser Parameter heißt "`Farbton"', "`Unbuntart"' (engl. \textit{hue} (H)). 
\end{itemize}
\begin{align*}
 40^\circ\ \text{entspricht dann } & \frac{1}{3} \cdot \mathrm{R} + \frac{2}{3} \cdot \mathrm{Y} \left(\frac{1}{3} \cdot 0^\circ
	+ \frac{2}{3} \cdot 60^\circ\right)\\
	&= \frac{1}{3} \cdot (1,0,0) + \frac{2}{3} \cdot (1,1,0) = \left(1, \frac{2}{3}, 0\right)
\end{align*}
Wir erhalte ein neues Farbsystem HSV

\subsection{Andere Farbsysteme}
\subsubsection{HSV-System}
\begin{description}
 \item[hue] (Farbton) $0^\circ \le H \le 360^\circ$
 \item[saturation] (Sättigung) $0 \le S \le 1$
 \item[value] ($\approx$ Helligkeit) $0 \le B \le 1$
\end{description}
$V = 1$ sind die Farben auf den drei Deckseiten des Würfels: mindest einer der drei Werte ist 1
\[max(r,g,b) = 1\]
\begin{center}
	\definecolor{azure}{rgb}{0.25,0.75,0.25}
 \psset{unit=4cm,xMin=0,xMax=1.5,yMin=0,yMax=1.5,zMin=0,zMax=1.5,Alpha=60,linecolor=gray,Beta=20,nameX=$r$,nameY=$g$,nameZ=$b$}
 \begin{pspicture}(-1,-0.5)(1,1.5)
  \pstThreeDCoor[linecolor=black]
  \pstThreeDBox(0,0,0)(1,0,0)(0,1,0)(0,0,1)
  \pstThreeDLine(0,0,0)(1,1,1)
  \pstThreeDSquare[fillstyle=vlines](1,0,0)(0,0,1)(0,1,0)
  \pstThreeDSquare[fillstyle=vlines](0,0,1)(1,0,0)(0,1,0)
  \pstThreeDSquare[fillstyle=vlines](0,1,0)(0,0,1)(1,0,0)
  \pstThreeDPut(1,1,1){W}
  \pstThreeDPut(0,0,0){K}
  \pstThreeDPut(1,0,0){R}
  \pstThreeDPut(1,1,0){Y}
  \pstThreeDPut(1,0,1){M}
  \pstThreeDPut(0,1,0){G}
  \pstThreeDPut(0,1,1){C}
  \pstThreeDPut(0,0,1){B}
 \end{pspicture}
\end{center}
Umrechnung HSV $\to$ RGB:
\begin{align*}
 \vthree{r''}{g''}{b''} &\text{sei die reine Farbe, die $H$ entspricht.}\\
 \vthree{r'}{g'}{b'} &= S \vthree{r''}{g''}{b''} + (1-S) \vthree{1}{1}{1}\\
 \text{Ergebnis } \vthree{r}{g}{b} &= V \vthree{r'}{g'}{b'}
\end{align*}
Umrechnung RGB $\to$ HSV
\begin{center}
 \psset{unit=3cm,Alpha=80,Beta=15}
 \begin{pspicture}(-0.2,-0.4)(1,1)
  \pstThreeDBox(0,0,0)(1,0,0)(0,1,0)(0,0,1)
  \pstThreeDSquare[fillstyle=vlines,hatchangle=0,hatchsep=1pt,hatchcolor=gray](0.7,0,0)(0,0,0.3)(0,0.3,0)
  \pstThreeDSquare[fillstyle=vlines,hatchangle=-60,hatchsep=1pt,hatchcolor=gray](0.7,0,0.3)(0.3,0,0)(0,0.3,0)
  \pstThreeDSquare[fillstyle=vlines,hatchangle=-60,hatchsep=1pt,hatchcolor=gray](0.7,0.3,0)(0,0,0.3)(0.3,0,0)
  \pstThreeDPut(1.04,-0.04,-0.04){$K$}
  \pstThreeDPut(-0.04,1.04,1.04){$W$}
  \pstThreeDPut(0.7,0.15,0.4){$V=0{,}3$}
 \end{pspicture}
 \hspace{2cm}
 \begin{pspicture}(-0.2,-0.4)(1,1)
  \pstThreeDBox(0,0,0)(1,0,0)(0,1,0)(0,0,1)
  \pstThreeDLine[fillstyle=vlines](1,0,0)(0.5,0,1)(0,1,1)(1,0,0)
  \pstThreeDPut(1.04,-0.04,-0.04){$K$}
  \pstThreeDPut(-0.04,1.04,1.04){$W$}
  \pstThreeDPut(0.5,1.4,0.5){$H$ konst.}
 \end{pspicture}\\
 \psset{unit=3cm,Alpha=100,Beta=20}
 \begin{pspicture}(-0.2,-0.4)(1,1.2)
  \pstThreeDLine[linecolor=gray](0.2,0.8,1)(1,0,0)(0,1,0.8)
  \pstThreeDLine[linecolor=gray](0.2,1,0.8)(1,0,0)(0,0.8,1)
  \pstThreeDLine[linecolor=gray](0,0.8,0.8)(1,0,0)(0.2,1,1)
  \pstThreeDSquare[linecolor=gray](0,0.8,0.8)(0,0,0.2)(0,0.2,0)
  \pstThreeDBox(0,1,0)(1,0,0)(0,-1,0)(0,0,1)
  \pstThreeDSquare[linecolor=gray](0,0.8,1)(0.2,0,0)(0,0.2,0)
  \pstThreeDSquare[linecolor=gray](0,1,0.8)(0.2,0,0)(0,0,0.2)
  \pstThreeDPut(1.04,-0.04,-0.04){$K$}
  \pstThreeDPut(-0.04,1.04,1.04){$W$}
  \pstThreeDPut(0.5,1.4,0.5){$S = 0{,}2$}
 \end{pspicture}
\end{center}
\begin{align*}
 V &= \max(r,g,b)\\
 \vthree{r'}{g'}{b'} = \vthree{r}{g}{b} \cdot \frac{1}{V}\\
 S &= 1 - \min(r',g',b')\\
 \rnode{v''}{\vthree{r''}{g''}{b''}} &= \left[\vthree{r'}{g'}{b'}-(1-S)\vthree{1}{1}{1}\right] \cdot \frac{1}{S}
\end{align*}
\rnode{rF}{reine Farbe} $\rightarrow$ Umwandlung in $H$ \nccurve[angleA=45]{->}{rF}{v''}
\begin{center}
 \psset{unit=3cm,Alpha=100,Beta=20}
  \begin{pspicture}(-0.2,-0.4)(1,1.6)
  \pstThreeDBox(0,1,0)(1,0,0)(0,-1,0)(0,0,1)
  \pstThreeDLine(0,1,1)(0.5,0,1)
  \pstThreeDDot(0,1,1)
  \pstThreeDDot(0.25,0.5,1)
  \pstThreeDDot(0.5,0,1)
  \pstThreeDPut(1.04,-0.04,-0.04){$K$}
  \pstThreeDPut(-0.04,1.04,1.04){$W$}
  \pstThreeDPut(0.25,0.5,1.4){$\vthree{r'}{g'}{b'}$}
 \end{pspicture}
\end{center}
$\vthree{r''}{g''}{b''}$ ist tatsächlich eine reine Farben
\begin{enumerate}
 \item Wir wissen $\max(r',g',b') = 1$ z. B. $r' = 1$
	\[r'' = [1-(1-S)1] \cdot \frac{1}{S} = S \frac{1}{S} = 1\]
 \item Nehmen wir nun an $(r',g',b') = g' = 1-S$
	\[g'' = [g'-(1-S)1] \cdot \frac{1}{S} = [-1 + S + 1 - S] \frac{1}{S} = 0\]
\end{enumerate}
Für $V = 0$ setze $S, H$ beliebig.\\
Für $V \neq 0$, $S = 0$, setze $H$ beliebig.
\begin{center}
 \psset{xunit=1.5,yunit=4.5}
 \begin{pspicture}(0,0)(6,2)
 \rput(0,-1){
  \psframe*[linecolor=red](0.3,2)(0.7,2.8)
  \psframe*[linecolor=green](1.3,2)(1.7,2.2)
  \psframe*[linecolor=blue](2.3,2)(2.7,2.5)
  \psaxes[labels=y,ticks=y](0,2)(0,3)(3,2)
  \uput{5pt}[-90](0.5,2){$r$}
  \uput{5pt}[-90](1.5,2){$g$}
  \uput{5pt}[-90](2.5,2){$b$}
  
  \psline{|<->|}(4.75,2.2)(4.75,2.8)\uput{1mm}[0](4.75,2.5){$S \cdot V$}
  \psline{|<->|}(4.75,2)(4.75,2.2)\uput{1mm}[0](4.75,2.1){$(1-S) \cdot V$}
  
  \psline[linestyle=dashed](3,2)(4.6,2)
  \psline[linestyle=dashed](3.7,2.8)(4.6,2.8)
  \psline[linestyle=dashed]{->}(0,2.8)(3.5,2.8)\uput*{1mm}[0](3.5,2.8){$\max \to V$}
  \psline[linestyle=dashed](3.7,2.2)(4.6,2.2)
  \psline[linestyle=dashed]{->}(0,2.2)(3.5,2.2)\uput*{1mm}[0](3.5,2.2){$\min$}
  \psline{|<->|}(-0.1,2.2)(-0.1,2.8)
  \pnode(-0.1,2.5){H_arrow}
  
  \psbrace(6,2.8)(6,3){schwarz = $1 - V$}
  \psbrace(6,2.2)(6,2.8){reine Farbe}
  \psbrace(6,2)(6,2.2){weiß = $V \cdot (1 - S)$}
 }
 \psset{yunit=0.6}
 \pnode(-0.3,0.95){H}
 \psaxes[labels=y,ticks=y](0,0)(0,0)(3,1)
 \psframe*[linecolor=red](0.3,0)(0.7,1)
 \psframe*[linecolor=green](1.3,0)(1.7,0)
 \psframe*[linecolor=blue](2.3,0)(2.7,0.5)
 
 \rput(4,0.8){\rnode{H_label}{\LARGE $H$}}
 \pnode(1.5,0.5){H_middle}
 
 \nccurve[angleA=-140,angleB=140]{->}{H_arrow}{H}
 
 \nccurve[angleA=-160,angleB=20]{->}{H_label}{H_middle}
 \end{pspicture}
\end{center}
Nachteil:
\begin{description}
 \item $\mathrm{R} = (1,0,0)$
 \item $\mathrm{Y} = (1,1,0)$
 \item $\mathrm{W} = (1,1,1)$
\end{description}
haben denselben $V$-Wert.

\subsubsection{HSL-System}
\begin{description}
 \item[hue] (Farbton) $0^\circ \le H \le 360^\circ$
 \item[saturation] (Sättigung) $0 \le S \le 1$
 \item[lightness] (oder "`luminance"') $L = \tfrac{1}{2}$ enthält das reine Farbensechseck und den Graupunkt
	$\left(\tfrac{1}{2},\tfrac{1}{2},\tfrac{1}{2}\right)$
\end{description}
\begin{center}
 \psset{Alpha=150,Beta=20}
 \begin{pspicture}(0,-1)(2,4)
  \pstThreeDLine[linewidth=2pt](0,0,2)(2,0,2)(2,0,0)(2,2,0)(0,2,0)(0,2,2)(0,0,2)
  \pstThreeDDot(1,1,1)
  \pstThreeDLine(1,1,1)(0,0,2)
  \pstThreeDLine(1,1,1)(2,0,2)
  \pstThreeDLine(1,1,1)(2,0,0)
  \pstThreeDLine(1,1,1)(2,2,0)
  \pstThreeDLine(1,1,1)(0,2,0)
  \pstThreeDLine(1,1,1)(0,2,2)
 \end{pspicture} \hspace{2cm}
 \begin{pspicture}(0,-1)(2,4)
  \pstThreeDLine[linewidth=2pt](1.5,0,0)(3,1.25,1.2)(3,3.75,0.8)(1.5,5,2)(0,3.75,0.8)(0,1.25,1.2)(1.5,0,0)
  \pstThreeDDot(1.5,2.5,1)
  \pstThreeDLine(1.5,2.5,1)(1.5,0,0)
  \pstThreeDLine(1.5,2.5,1)(3,1.25,1.2)
  \pstThreeDLine(1.5,2.5,1)(3,3.75,0.8)
  \pstThreeDLine(1.5,2.5,1)(1.5,5,2)
  \pstThreeDLine(1.5,2.5,1)(0,3.75,0.8)
  \pstThreeDLine(1.5,2.5,1)(0,1.25,1.2)
 \end{pspicture} \hspace{4cm}
 \begin{pspicture}(-1,-2)(1,2)
  \psset{linecolor=gray}
  \pspolygon(0,2)(-1,0)(0,-2)(1,0)
  \psellipse(0,0)(1,0.5)
  \psline(0,2)(0,-2)
  \psdot(0,0)
  \uput{3pt}[45](0,0){$\left(\tfrac{1}{2},\tfrac{1}{2},\tfrac{1}{2}\right)$}
  \uput{3pt}[180](-1,0){$L=\frac{1}{2}$}
  \uput{3pt}[180](0,2){$L=1$}
  \uput{3pt}[180](0,-2){$L=-1$}
 \end{pspicture}
\end{center}
Umrechnung HSL $\to$ RGB:
\begin{align*}
 \vthree{r''}{g''}{b''} &\text{sei die reine Farbe, die $H$ entspricht.}\\
 \vthree{r'}{g'}{b'} &= S \vthree{r''}{g''}{b''} + (1-S) \vthree{1}{1}{1}\\
 \text{Für } 0 \le L \le \frac{1}{2}: \vthree{r}{g}{b} &= \vthree{r'}{g'}{b'} \cdot 2L\\
 \text{Für } \frac{1}{2} \le L \le 1: \vthree{r}{g}{b} &= \vthree{r'}{g'}{b'} \cdot (2-2L) +
			\vthree{1}{1}{1} \cdot (2L - 1)
\end{align*}
\begin{center}
 \psset{unit=3cm,Alpha=200,Beta=10}
  \begin{pspicture}(-0.2,-0.2)(1,1.4)
  \pstThreeDLine[fillstyle=solid,fillcolor=gray](0.15,0.15,0.15)(0.3,0,0)(0.3,0.3,0)(0.15,0.15,0.15)
  \pstThreeDLine[fillstyle=solid,fillcolor=gray](0.15,0.15,0.15)(0.3,0.3,0)(0,0.3,0)(0.15,0.15,0.15)
  \pstThreeDLine[fillstyle=solid,fillcolor=gray](0.15,0.15,0.15)(0,0.3,0)(0,0.3,0.3)(0.15,0.15,0.15)
  \pstThreeDLine[fillstyle=solid,fillcolor=gray](0.15,0.15,0.15)(0,0.3,0.3)(0,0,0.3)(0.15,0.15,0.15)
  \pstThreeDLine[fillstyle=solid,fillcolor=gray](0.15,0.15,0.15)(0,0,0.3)(0.3,0,0.3)(0.15,0.15,0.15)
  \pstThreeDLine[fillstyle=solid,fillcolor=gray](0.15,0.15,0.15)(0.3,0,0.3)(0.3,0,0)(0.15,0.15,0.15)
  \pstThreeDDot(0.15,0.15,0.15)
  \pstThreeDBox(1,1,0)(-1,0,0)(0,-1,0)(0,0,1)
  \pstThreeDPut(-0.1,-0.1,-0.1){$K$}
  \pstThreeDPut(1.1,1.1,1.1){$W$}
 \end{pspicture}
 \hspace{2cm}
 \begin{pspicture}(-0.2,-0.2)(1,1.4)
  \pstThreeDLine[linecolor=gray](0.6,0.4,0.4)(0.6,0.6,0.4)(0.4,0.6,0.4)(0.4,0.6,0.6)(0.4,0.4,0.6)(0.6,0.4,0.6)(0.6,0.4,0.4)
  \pstThreeDLine[linecolor=gray](0,0,0)(0.6,0.4,0.4)(1,1,1)(0.4,0.6,0.6)(0,0,0)
  \pstThreeDLine[linecolor=gray](0,0,0)(0.6,0.6,0.4)(1,1,1)(0.4,0.4,0.6)(0,0,0)
  \pstThreeDLine[linecolor=gray](0,0,0)(0.4,0.6,0.4)(1,1,1)(0.6,0.4,0.6)(0,0,0)
  \pstThreeDBox(1,1,0)(-1,0,0)(0,-1,0)(0,0,1)
  \pstThreeDPut(-0.1,-0.1,-0.1){$K$}
  \pstThreeDPut(1.1,1.1,1.1){$W$}
  \pstThreeDPut(1.6,0,0.5){$S = 0{,}2$ in HSL}
 \end{pspicture}
\end{center}
$S = 1$ Farben auf allen 6 Seitenflächen des Würfels

\subsection{YCrCb/YPrPb-Farbsystem}
Einsatz beim Farbfernsehen
\begin{align*}
 Y &= 0{,}299 \cdot R + 0{,}587 \cdot G + 0{,}114 \cdot B \qquad \text{(Helligkeit)}\\[1em]
 Cr &= \Rnode[vref=10pt,href=1.1]{Cr}{...} \\[1em]
 Cb &= \Rnode[vref=-2pt,href=1.1]{Cb}{...}
\end{align*}
\psbrace(Cb)(Cr){lineare Ausdrücke in $R$, $G$, $B$ (Chromanz, Farbigkeit)}
\begin{align*}
 0 &\le R, G, B \le 1\\
 0 &\le Y \le 1
 -\frac{1}{2} \le Cr, Cb \le +\frac{1}{2}
\end{align*}
\begin{center}
	% 10.1
	\psset{unit=3cm,xMin=-0.4,xMax=1.3,yMin=-0.9,yMax=1.3,zMin=-0.1,zMax=1.1,Alpha=70,Beta=15,
		nameX=$r$,nameY=$g$,nameZ=$b$}
	\begin{pspicture}(-0.2,-0.2)(1,1.4)
	\pstThreeDCoor[linecolor=black]
	\pstThreeDBox(0,0,0)(1,0,0)(0,1,0)(0,0,1)
	\pstThreeDLine{->}(0,0,0)(0.611,1.2,0.233)
	\pstThreeDLine{->}(0,0,0)(-0.327,-0.662,1)
	\pstThreeDLine{->}(0,0,0)(1,-0.827,-0.083)
	\pstThreeDPut(0.611,1.3,0.233){$Y$}
	\pstThreeDPut(1.15,-0.827,-0.083){$Cr$}
	\pstThreeDPut(-0.327,-0.662,1.1){$Cb$}
	\pstThreeDDot[drawCoor=true](0.509,1,0.194)
	\pstThreeDDot[drawCoor=true,dotstyle=none](-0.327,-0.662,1)
	\pstThreeDDot[drawCoor=true,dotstyle=none](1,-0.827,-0.083)
	\end{pspicture}
\end{center}

\section{Subtraktive Farbmischung (z. B. beim Drucken)}
3 Grundfarben
\begin{description}
 \item $C = $ cyan, $B, G$ wird durchgelassen, $R$ wird absorbiert
 \item $M = $ magenta, $B, R$ wird durchgelassen, $G$ wird absorbiert
 \item $Y = $ gelb, $R, G$ wird durchgelassen, $B$ wird absorbiert
\end{description}
$C + M$ nur Blau bleibt übrig\\
$C + M + Y = $ schwarz

\subsection{CMY-System}
$0 \le C, M, Y \le 1$
\begin{align*}
 C &:= 1 - R\\
 M &:= 1 - G\\
 Y &:= 1 - B
\end{align*}
\begin{center}
  \psset{xunit=1.5,yunit=3}
 \begin{pspicture}(0,0)(3,1)
  \psframe*[linecolor=red](0.3,0)(0.7,0.6)
  \psframe*[linecolor=green](1.3,0)(1.7,0.8)
  \psframe*[linecolor=blue](2.3,0)(2.7,0.5)
  \psaxes[labels=y,ticks=y](0,0)(0,1)(3,0)
  \uput{5pt}[-90](0.5,0){$R$}
  \uput{5pt}[-90](1.5,0){$G$}
  \uput{5pt}[-90](2.5,0){$B$}  
 \end{pspicture}
 \hspace{2cm}
 \begin{pspicture}(0,0)(3,1)
  \psframe*[linecolor=cyan](0.3,0)(0.7,0.4)
  \psframe*[linecolor=magenta](1.3,0)(1.7,0.2)
  \psframe*[linecolor=yellow](2.3,0)(2.7,0.5)
  \psaxes[labels=y,ticks=y](0,0)(0,1)(3,0)
  \uput{5pt}[-90](0.5,0){$C$}
  \uput{5pt}[-90](1.5,0){$M$}
  \uput{5pt}[-90](2.5,0){$Y$}  
 \end{pspicture}
\end{center}

\subsection{CMYK-System (Vierfarbdruck)}
Zusätzlich $K = $schwarz. (Das Schwarz von $K$ wird dunkler als von $C + M + Y$ oder um Druckfarbe zu sparen.)
\begin{align*}
 K' &:= \min(C, M, Y)\\
 C' &:= C - K'\\
 M' &:= M - K'\\
 Y' &:= Y - K'
\end{align*}
(möglichst viel Farbe durch $K$ ersetzen)

\section{Digitale Farbdarstellung}
heutzutage:
\begin{itemize}
 \item 8 Bit pro Farbkanal $(r,g,b)$ ... 24 Bit pro Bildpunkt\\
	$\Rightarrow 2^{24} = 16\ \text{Mio. Farben}$ (True Color)
 \item 4-Kanal-Darstellung $(r, g, b, \alpha)$
	$\alpha$ ist für Transparenz:
	\begin{itemize}
	 \item $\alpha = 0...$ durchsichtig; Farbe wird vom Hintergrund genommen,
	 \item $\alpha = 1...$ Farbe wird von $(r, g, b)$ bestimmt
	 \item $0 < \alpha < 1...$ teilweise transparent
	\end{itemize}
	$\rightarrow$ 32 bit pro Pixel
\end{itemize}
