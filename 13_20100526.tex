\section{Darstellung einer Szene durch Überstreichen (Scanline-Algorithmus)}
\paragraph*{Vereinfachte Annahme} Szene besteht auzs Dreiecken.
\begin{itemize}
 \item Der Algorithums beruht auf der gleichen Idee wie der Algorithmus zum Ausfüllen gerasterter Flächen
	(s. Abschnitt \ref{sec:fill})
	\begin{itemize}
	 \item Überstreiche Szenee mit horizontalen Geraden (von unten nach oben, mit steigendem $y$)
		\begin{itemize}
		 \item Für jede Rasterzeile gehe Pixel von links nach rechts durch (mit steigendem $x$)
		\end{itemize}
	\end{itemize}
 \item Prinzip: Der Algorithmus aus Abschnitt \ref{sec:fill} wird für alle Dreiecke parallel angewendet
\end{itemize}
\begin{center}
 \begin{pspicture}(0,0)(5,4)
  \psframe(0,0)(5,4)
  \rput[l](0,0.2){\psline(0,0)(5,0)}
  \rput[l](0,0.4){\psline(0,0)(5,0)}
  \rput[l](0,0.6){\psline(0,0)(5,0)}
  \rput[l](0,0.8){\psline(0,0)(5,0)}
  \rput[l](0,2.7){\psline(0,0)(5,0)}
  \pspolygon(3.5,0.6)(4,3)(1,1)
  \pspolygon(0.5,1.4)(1,3.2)(4.5,0.9)
 \end{pspicture}
\end{center}
\begin{description}
\item[Vorverarbeitung] Wir erstellen eine Kantenliste (Edge List, EL), sortiert nach den Koordinaten des unteren Endpunktes der Kante,
	zuerst nach der $y$-Koordinate sortiert, bei gleicher $y$-Koordinate nach $x$-Koordinate
	(horizontal Kanten brauchen nicht gespeichert werden).
\item[Bearbeitung einer Zeile] Wir benötigen zur Abarbeitung einer Zeile die Menge der von der Zeile
	geschnittenen Dreiecke und für diese dann die nötigen Parameter, wie sie für die Fülle-Zeile-Funktion aus
	Abschnitt \ref{sec:fill} benutzt wurden.

	Wir verwalten die Daten in einer \emph{"`Aktive-Kanten-Liste"'} (AEL). Darin enthalten sind alle Kanten, die
	die aktuelle Scanline schneiden. Sie sind sortiert nach der $x$-Koordinate des Schnittpunktes mit der Scanline.
	Zu jeder Kante speichern wir die Informationen, die wir benötigen. Um das Dreieck für diese Zeile zu rastern und
	die entsprechenden Werte für die Zeile auszurechnen.

	z. B.
	\begin{itemize}
	\item zugehöriges Dreieck
	\item linke oder rechte Begrenzung
	\item $y$-Koordinate des oberen Endpunktes (um die Kante au der AEL entfernen zu können)
	\item $x$-Koordinate Schnittpunktes mit der Scanline (entspricht $x_{\mathrm{links}}$ bzw. $x_{\mathrm{rechts}}$)
	\item $\Delta x$ (entspricht $\Delta x_{\mathrm{links}}$ bzw. $\Delta x_{\mathrm{rechts}}$)
	\end{itemize}
	die folgenden Daten brauchen nur in einer linken Begrenzung gespeichert werden:
	\begin{itemize}
	 \item $z_{\mathrm{links}}, I_{\mathrm{links}}, \Delta z_{\mathrm{links}}, \Delta I_{\mathrm{links}}, \Delta z,
		\Delta I$ (Notation s. Abschnitt \ref{sec:fill})
	\end{itemize}
	und entsprechende Werte für andere zu interpolierende Daten.
\item[Änderung der AEL beim Übergang von $\boldsymbol{y}$ auf $\boldsymbol{y+1}$] wie bei der äußeren Schleife in
	Abschnitt \ref{sec:fill}.
	\begin{itemize}
	 \item inkrementiere $x, z_{\mathrm{links}}, I_{\mathrm{links}}$ um $\Delta x, \Delta z_{\mathrm{links}},
		\Delta I_{\mathrm{links}}$
	\end{itemize}
	Außerdem kann es nötig sein Kanten aus der Liste zu löschen (Kante endet zwischen $y$ und $y+1$),
	neue Kanten aufzunehmen (Kante beginnt zwischen $y$ und $y+1$) oder die Reihenfolge der Kanten zu ändern
	(Kanten schneiden sich zwischen $y$ und $y+1$)
	\begin{enumerate}
	 \newcommand{\bs}{\boldsymbol}
	 \item \textbf{Kante endet zwischen $\bs y$ und $\bs{y+1}$}\\
		Vergleiche $y$-Koordinate des oberen Endpunktes einer Kante in der AEL mit der aktuellen
		Scanline-Koordinate
	 \item \textbf{Kante beginnt zwischen $\bs y$ und $\bs{y+1}$}\\
		Vergleiche $y$-Koordinate des unteren Endpunktes der ersten noch nicht verwendeten Kante in der EL
		mit aktuellen Scanline-Koordinaten.
	 \item \textbf{Kanten schneiden sich zwischen $\bs y$ und $\bs{y+1}$}\\
		prinzipiell vergleiche die neuen $x$-Werte zweier Kanten die in der alten AEL benachbart waren
		\paragraph*{Problem} Die Scanline wird nicht Kontinierlich
		\begin{center}
		 \begin{pspicture}(0,0)(5,1)
		  \psline(0,0)(5,0)\uput{3pt}[180](0,0){$y$}
		  \psline(0,1)(5,1)\uput{3pt}[180](0,1){$y+1$}
		  \psline(1,1)(4,0)
		  \psline(1.2,0)(2.5,1)
		  \psline(1.4,0)(1.2,1)
		  \psline(1.5,0)(1.8,1)
		  \psline(2.3,0)(2,1)
		  \psline(2.7,0)(2.3,1)
		  \psline(2.9,0)(2.87,1)
		  \psline(3.1,0)(3.13,1)
		  \psline(3.2,1)(4.2,0)
		  \psline(1.4,0)(2.7,1)
		  \psline(1.6,0)(2.9,1)
		  \psline(1.9,0)(2.9,1)
		  \psline(2.1,0)(3.4,1)
		 \end{pspicture}
		\end{center}
		Es können sich auch nichtbenachbarte Kanzen zwischen $y$ und $y+1$ schneiden, wenn die dazwischen
		liegenden Kanten ebenfalls geschnitten werden. Wir müssen also nicht nur bei $y$ benachbarte Kanten
		vergleichen, sondern auch die nache einer Kreuzung entstehenden neuen Nachbarn.

		Dieser Ansatz läuft auf ein Buble-Sort- oder Insertion-Sort-Prinzip hinaus. Dann ist Neusortieren
		mit effizienten Sortierverfahren.
	\end{enumerate}
\item[Füllen einer Zeile] Zeile wird von links nach rechts durchlaufen. Beachten wir dabei die Tiefeninformation
	entspricht dies einem Überstreichen einer Ebene $y = \mathbf{const}$ mit einer Geraden $x = \mathbf{const}$
	\begin{center}
	 \psset{unit=1.5}
	 \begin{pspicture}(0,-0.2)(4,2.2)
	  \psaxes[ticks=none]{->}(0,0)(0,0)(4,2)
	  \uput{3pt}[135](0,2){$z$}
	  \uput{3pt}[-45](4,0){$x$}
	  \psline{*-*}(1,1.75)(3,0.5)
	  \uput{3pt}[90](1,1.75){Kante $AB$}
	  \uput{3pt}[-45](3,0.5){Kante $AC$}
	  \rput[tr](2,1.125){
		\begin{minipage}{2.7cm}
			Schnitt mit dem Dreieck $ABC$
		\end{minipage}
	   }
	  \psline{*-*}(2.5,1.125)(3.5,2)
	  \uput{3pt}[-45](2.5,1.125){Kante $DE$}
	  \uput{3pt}[90](3.5,2){Kante $DF$}
	  \rput[tl](3,1.5025){Schnitt mit dem Dreieck $DEF$}
	 \end{pspicture}
	\end{center}
	in dieser Ebene haben wir Kante, die den Schnitten von Dreiecken mit der Ebene entsprechen. Der Dreiecksschnitt,
	dessen $z$-Koordinate für ein gegebenes $x$ am kleinsten ist, bestimmt die Färbung des Pixels.

	Analog zur AEL legen wir eine aktive Dreiecksliste (ATL, auch active span list) an, om der die von der aktiven
	Geraden $x = \mathbf{const}$ geschnittenen Dreiecksschnitte nach $z$-Koordinate sortiert sind und die zugehörigen
	Werte für die Interpolation gespeichert sind. Das sind $z, I$ und $\Delta z$, sowie $\Delta I$ und $x_{\mathrm{rechts}}$.

	Modifizierung der ATL beim Übergang von $x$ auf $x+1$
	\begin{enumerate}
	 \item Dreieck einfügen: $x_{\mathrm{links}}$ liegt zwischen $x$ und $x+1$ für eine linke
		Kante in der AEL
	 \item Dreieck löschen: $x_{\mathrm{rechts}}$ liegt zwischen $x$ und $x+1$
	 \item Zwei Dreiecke schneiden sich ($z$-Werte tauschen Reihenfolge)
	\end{enumerate}
	zu zeichnen ist die aktuelle Farbe des Dreiecks, das in der ATL aktuell an der ersten Stelle steht.
\end{description}
\subsubsection{Varianten, Sonderfälle, ...}
\begin{enumerate}
 \item Es ist häufig sinnvoll, sobald zwei Kanten in der AEL benachbart werden, den Schnittpunkt zu berecnhen und
	diesen, falls er existiert, in einem \emph{Event-Schedule} einzufügen. Dieser verwaltet die "`Zeit"'-punkte,
	an denen sich die Reihenfolge der Kanten ändert in einer Prioritätswarteschlange.
 \item Vereinfachungen bei sich nicht schneidenden Dreiecken
 \item Kombination mit dem Tiefenpuffer\\
	z. B. tiefenpuffer für eine Zeile des Bildes
	\begin{itemize}
	 \item Überstreiche Szene zeilenweise
	 \item Vor jedem Wechsel in eine neue Zeile leere Tiefenpuffer
	 \item für jede Zeile: Zeichne alle Dreiecke in der aktuellen AEL (in beliebiger Reihenfolge) nach dem
		Tiefenpuffer-Prinzip
	\end{itemize}
	Analog: Tiefenpuffer für ein Pixel
	\begin{center}
	 \begin{tabular}{p{.45\linewidth}|p{.45\linewidth}}
	  Größe des Tiefenpuffers & Struktur des Algorithmus\\
	  \hline\hline
		Breite $\times$ Höhe $\times$ Genauigkeit &
		\begin{minipage}{\linewidth}
		 \vspace{1em}
		 \begin{verbatim}
Für alle Dreiecke
    Für alle Zeilen
        Für alle Spalten
		 \end{verbatim}
		\end{minipage}\\
		\hline
		Breite $\times$ Genauigkeit &
		\begin{minipage}{\linewidth}
		 \vspace{1em}
		 \begin{verbatim}
Für alle Zeilen
    Für alle Dreiecke
        Für alle Spalten
		 \end{verbatim}
		\end{minipage}\\
		\hline
		Genauigkeit &
		\begin{minipage}{\linewidth}
		 \vspace{1em}
		 \begin{verbatim}
Für alle Zeilen
    Für alle Spalten
        Für alle Dreiecke
		 \end{verbatim}
		\end{minipage}\\
	 \end{tabular}
	\end{center}
\end{enumerate}




