Bilder im KI-Heft
\section{Bézierkurven (Bézier-Splines)}
\begin{itemize}
 \item Ordnung = $d$ = Grad der Polynome
 \item $d + 1$ Kontrollpukte $\quad \vec P_0, \vec P_1, ..., \vec P_d$
 \item \textsc{Bernstein}polynome:
	\[\boxed{B^d_i = \binom{d}{i} \cdot t^i \cdot (1-t)^{d-i}}\qquad i = 0,1,...,d \qquad 0 \le t \le 1\]
%	\begin{tabular}{rp{0.3\linewidth}p{0.3\linewidth}}
	 $d = 1$ %&\begin{minipage}{\linewidth}
		\begin{align*}
		 B_0^1(t) &= 1 \cdot t^0 (1-t)^1 = 1-t\\
		 B_1^1(t) &= \binom{1}{1} \cdot t^1 \cdot (1-t)^0 = t
		\end{align*}
%		\end{minipage}
%		&
%		21.1
	 $d = 2$ %&\begin{minipage}{\linewidth}
		\begin{align*}
		 B_0^2(t) &= \binom{2}{0} \cdot t^0 (1-t)^2 = (1-t)^2 = 1-2t+t^2\\
		 B_1^2(t) &= \binom{2}{1} \cdot t^1 \cdot (1-t) = 2t(1-t) = 2t-2t^2\\
		 B_2^2(t) &= \binom{2}{2} \cdot t^2 \cdot (1-t)^0 = t^2
		\end{align*}
%		\end{minipage}
%		&
		21.2
	 $d = 3$ %&\begin{minipage}{\linewidth}
		\begin{align*}
		B_0^3(t) &= (1-t)^3\\
		 B_1^2(t) &= \binom{3}{1} \cdot t^1 \cdot (1-t)^2 = 3t(1-t) = 3t(1-t)^2\\
		 B_2^2(t) &= \binom{3}{2} \cdot t^2 \cdot (1-t) = 3t^2(1-t) = 3t^2 - 3t^3\\
		 B_3^3(t) &= t^3
		\end{align*}
%		\end{minipage}
%		&
		21.3	
%	\end{tabular}
	Eigenschaften der Bernsteinpolynome:
	\begin{enumerate}
	 \item $B_i^d(t) \ge 0$ für $0 \le t \le 1$
	 \item $\boxed{\sum\limits_{i=0}^{d} B_i^d(t) = 1}$
		\Bew	\[\sum\limits_{i=0}^{d} \binom{d}{i} t^i (1-t)^{d-i} = [t + (1-t)]^d = 1^d = 1\]
	 \item Rekursion:
		\begin{align*}
		\rnode{left}{B_i^d(t)} &= \rnode{right}{t \cdot B_{i-1}^{d-1}(t) + (1-t)B_i^{d-1}(t)} \qquad 1 \le i \le d - 1\\
		 B_d^d(t) &= t \cdot B_{d-1}^{d-1}(t)	= t^d \\
		 B_0^d(t) &= t \cdot B_{d-1}^{d-1}(t) + (1-t) B_0^{d-1}(t) = (1-t)
		\end{align*}
		\ncbox{left}{right}
		\Bew (von (1))
			\begin{align*}
			 & \qquad t \cdot \binom{d-1}{i-1} \cdot t^{i-1} (1-t)^{(d-1)-(i-1)} + (1-t) \binom{d-1}{i} \cdot
				t^{i} (1-t)^{d-1-i}\\
			 &= \binom{d-1}{i-1}t^i (1-t)^{d-i} + \binom{d-1}{i} \cdot t^i (1-t)^{d-i}
				= t^i (1-t)^{d-i} \underbrace{\left[\binom{d-1}{i-1}+ \binom{d-1}{i}\right]}_{\binom{d}{i}}]
			\end{align*}
	\end{enumerate}
	\paragraph*{Bemerkung} $(B_0^d(t), ..., B_d^d(t))$ sind die Wahrscheinlichkeiten bei der Binomialverteilung mit
	Er"-folgs"-wahr"-schein"-lich"-keit $p = t$\\[1em]
	$B_i^d(p) =$ Wahrscheinlichkeit, dass bei $d$ Wiederholungen genau $i$ Erfolge auftreten.
\end{itemize}
$\Rightarrow$ Splinekurve $\vec P(t)$:
\[
 \boxed{\vec P(t) = B_0^d \cdot \vec P_0 + B_1^d(t) \vec P_1 + ... + B_d^d(t) \cdot \vec P_d} \qquad 0 \le t \le 1
\]
$d = 1$:
\[\vec P(t) = (1-t) \cdot \vec P_0 + t \vec P_1 = \vec P_0 + t(\vec P_1 - \vec P_0)\]
\begin{center}
 21.4
\end{center}
Eigenschaften:
\begin{enumerate}
 \item Kurve beginnt bei $P_0$ und endet bei $P_d$. Die übrigen Punkte werden im allgemeinen nicht durchlaufen.
 \item die Tangentenrichtung am Anfang und am Ende ist $\overrightarrow{P_0P_1}$ bzw.
	$\overrightarrow{P_{d-1}P_d}$ (Beweis s. Übung, Aufgabe 29).
 \item Die Kurbe verläuft in der Konvexen Hülle der Kontrollpunkte.
	\begin{center}
	 21.5
	\end{center}
 \item Die Punkte $\vec P_0, ..., \vec P_d$ können $\in \mathbb{R}^2, \in \mathbb{R}^3, ..., \mathbb{R}^n$ sein.
	Bézierkurven \emph{im Raum} werden auf die gleiche Art definiert.
 \item Verminderung der Variation:
	\begin{center}
	 \begin{tabular}{|p{0.97\linewidth}|}
	  \hline
	  Eine Gerade in der Ebene (eine Ebene im Raum) schneidet die Bézierkurve höchstens so oft wie das
	  Kontrollpolygon.\\
	  \hline
	 \end{tabular}\\
	 21.5
	\end{center}
\subsection{Der Algorithmus von de Casteljau}
zur Berechnung von $P(t)$
\begin{center}
 21.7\\
 21.8
\end{center}
$O(d^2)$ Operationen
\[\mathcal B^d(t)(P_0,...,P_d) := \sum\limits_{i=0}^d B_i^d(t) \cdot P_i\]
\end{enumerate}
\Bew des Casteljau-Algorithmus:
\begin{align*}
 \mathcal B^d(t)(P_0,...,P_d) &\stackrel{?}= (1-t) \mathcal B^{d-1}(P_0, ..., P_{d-1})
	+ t \cdot \mathcal B^{d-1}(P_1, ..., P_d)\\
	&= (1-t) \sum\limits_{i=0}^{d-1} B_i^{d-1}(t) P_i + t \cdot <-- ?
		 \underbrace{\sum\limits_{i=0}^{d-1} B_i^{d-1}(t) P_{i+1}}_
			{=\sum\limits_{j=1}^{d} B_{j-1}^{d-1}(t) P_j}\\
	&= (1-t) \sum\limits_{j=0}^{d-1} B_j^{d-1}(t) P_{j} + t \cdot \sum\limits_{j=1}^{d} B_{j-1}^{d-1}(t) P_j\\
	&= \text{(*) verwenden}
\end{align*}
\subsubsection{Algorithmus}
\begin{description}
 \item[Eingabe] $P_0, P_1, ..., P_d = P_0^0, P_1^0, ..., P_d^0 \qquad 0 < t < 1$
 \item[Ausgabe] $P(t)$
\end{description}
\begin{lstlisting}[morekeywords=to]
for $l := 1$ to $d$
	for $i := 0$ to $d-l$
		$P_i^l := (1-t) P_i^{l-1} + t P_{i+1}^{l-1}$
return $P_0^d = P(t)$
\end{lstlisting}

\subsubsection{Zerlegung von Bézierkurven}
\begin{itemize}
 \item $P_0, P_0^1, P_0^2, ..., P_0^d$ bilde das Kontrollpolygon für die kurve $P(x), 0 \le x \le t$
 \item $P_0^d, P_1^{d-1}, ..., P_{d-1}^1, P_d^0$ bilden das Kontrollpolygon für die Kurve $P(x), t \le x \le 1$
 \item häufigster Fall: $t = \dfrac{1}{2}$
\end{itemize}


