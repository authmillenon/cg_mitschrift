\subsection{Uniforme kubische B-Splines}
\textbf{Spezialfall:} $u_0 = 0 \quad u_1 = 1 \quad u_2 = 2 ... u_i = i$ (uniform)
\[C^2_i(t) = C^2_0(t-i)\]
\begin{center}
 %23.1
\end{center}
Die Funktionen $C_1^k, C_2^k, C_3^k, ...$ ergeben sich au $C_0^k$ durch verschieben nach rechts um $i$
(Beweis durch vollständige Induktion nach $k$)
\[\boxed{C_i^k(t) = C_0^k(t-i) \qquad \forall i, \forall k}\]
Es reicht, $C_0^k$ zu bestimmen
\begin{align*}
 d=1{:}\qquad C_0^2(t) &= \frac{t}{1} \cdot \underbrace{C_0^1(t)}_{1 \text{ für } t \in [0,1]} +
				\frac{2-t}{1} \underbrace{C_1^1(t)}_{1 \text{ für } t \in [1,2]}\\
		&= \begin{cases}
		    t, & 0 \le t \le 1\\
		    2-t, & 1 \le t \le 2\\
		    0
		   \end{cases}	% 23.2
 d=2{:}\qquad C_0^3(t) &= \frac{t}{2} %23.3
\end{align*}
...
\begin{align*}
 b_1(s) &= \frac{s}{2} a_1(s) = \frac{s}{2} \cdot s = \frac{s^2}{2}\\
 b_2(s) &= \frac{s+1}{2} a_2(s) + \frac{2-s}{2} a_1(2) = -\left(\frac{1}{2} - \frac{1}{2}\right)^2 + \frac{3}{4}\\
 b_3(s) &= \frac{1-s}{2} a_2(s) = \frac{(1-s)^2}{2}
\end{align*}
\begin{center}
 % 23.4
\end{center}
\begin{align*}
 X(t) &= % 23.5
\end{align*}
\[X(t) = b_1(s) \cdot P_l + b_2(s) P_{l-1} + b_3(s) P_{l-2}\]
\begin{center}
 % 23.6
\end{center}
\begin{align*}
 s &= t - \lfloor t \rfloor\\
 0 &\le s < 1\\
 1 &= b_1(s) + b_2(s) + b_3(s)
\end{align*}
% 23.7 d = 3
\[X(t) = d_1(s) \cdot P_l + d_2(s) P_{l-1} + d_3(s) P_{l-2} + d_4(s) P_{l-3}\]
\begin{center}
 % 23.8
\end{center}
\begin{align*}
 s &= t - \lfloor t \rfloor\\
 0 &\le s < 1\\
 1 &= d_1(s) + d_2(s) + d_3(s) + d_4(s)
\end{align*}
\Satz Kubische Splines sind $2 \times$-stetig differenzierbar
\begin{align*}
 d_1'(0) &= d_1''(0) = 0 = d_2(0) = 0 & d_1'(s) &= \frac{s^2}{2} & d_1''(s) &= s\\
 \frac{1}{6} &= d_1(1) = d_2(0) = \frac{1}{6} \quad \checkmark	& d_2'(s) &= \frac{3 + 6s +9s^2}{6}\\
 \frac{1}{2} &= d_1'(1) = d_2'(0) = \frac{1}{2} \quad \checkmark	& d_2''(s) &= \frac{6 + 18s^2}{6}\\
 1 &= d_1''(1) = d_2''(0) = 1 \quad \checkmark
\end{align*}
genauso
\begin{align*}
 d_2(1) &= d_3(0) & d_3(1) = 0\\
 d_2'(1) &= d_3'(0) & d_3'(1) = 0\\
 d_2'(1) &= d_3'(0) & d_3''(1) = 0\\
\end{align*}

\subsection{Rationale B-Splines}
\Satz B-Splines sind invariant unter affinen Transformationen: Wenn das Kontrollpolygon transformiert wird, wird auch
	die Kurve transformiert
\begin{center}
 % 23.9
\end{center}
\Defi \textbf{Ratio nale B-Splines} sind B-Splines in homogenen Koordinaten.\\[1em]
z. B. in der Ebene
\begin{align*}
 x(t) &= f(t)	& 0 &\le t \le l
 y(t) &= g(t)	& 0 &\le t \le l
 w(t) &= h(t)	& 0 &\le t \le l
\end{align*}
$f, g, h$ sind stückweise Polynomfunktionen
\[w(t)> 9\]
kartesische Koordinaten:
\begin{align*}
 x(t) &= \rnode{x}{\frac{f(t)}{h(t)}} & \rnode{sf}{\text{Stückweise rationale Funktionen}}\\
 y(t) &= \rnode{y}{\frac{g(t)}{h(t)}}
\end{align*}
\ncline{->}{sf}{x}
\ncline{->}{sf}{y}
Mit rationalen B-Spline kann man z. B. einen Kreis zeichnen.
\begin{align*}
 x(t) &= \frac{2t}{1+t^2}\\
 y(t) &= \frac{1-t^2}{1+t^2}\\
\end{align*}
\begin{center}
 % 23.10
\end{center}
\Satz Rationale Splines sind invarianz unter projektiven Transformationen.\\[1em]
Bildliche Interpretation:
\begin{center}
 % 23.11
\end{center}
\hspace{8cm}\rnode{w}{"`Gewichte"'}
\begin{align*}
 P_0 &= \left(x_0, y_0, \rnode{w_0}{w_0}\right)\\
 P_1 &= \left(x_1, y_1, w_1\right)\\
 P_2 &= \left(x_2, y_2, w_2\right)\\
	&= \quad \vdots
\end{align*}
\ncline{->}{w}{w_0}

\subsubsection[NURBS]{NURBS (non-uniform rational B-Splines)}
"`non-rational"' B-Splines = "`gewöhnliche"' B-Splines mit Polynomfunktionen.