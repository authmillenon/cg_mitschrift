\subsubsection{Rationale Splines und Gewichte}
\[
 \begin{array}{rr|l}
      & \textbf{kartesische Koordinaten} & \textbf{Gewicht}\\
  \hline
  P_0 & (x_1 \quad y_1 \quad (z_1)) & w_1\\
  P_1 & (x_2 \quad y_2 \quad (z_2)) & w_2\\
	& \vdots & w_3
 \end{array}
\]
Als Kontrollpolygon für dien rationalen Spline verwendet man
\begin{center}
 $(w_1x_1,w_1y_1,w_1)$\\
 $(w_2x_2,w_2y_2,w_2)$\\
 $\vdots$
\end{center}

\paragraph*{Beispiel} quadratische nicht rationaler B-Spline
\begin{center}
%  24.1
\end{center}
\emph{rationaler Spline} mit Gewichten
\begin{align*}
 X(t) &= \frac{1}{8} \vtwo{w_4 P_4}{w_4} + \frac{3}{4} \vtwo{w_5 P_5}{w_5} + \frac{1}{8} \vtwo{w_6 P_6}{w_6} \\
      &= \left(\frac{1}{8} w_4\right) P_4 + \left(\frac{3}{4} w_5\right) P_5 + \left(\frac{1}{8} w_6\right) P_6\\
      \hline
      &= \frac{1}{8} w_4 + \frac{3}{4} w_5 + \frac{1}{8} w_6
\end{align*}
Ein gewichtetes Mittel der 3 Punkte $P_4, P_5, P_6$, Gewichte: $\frac{1}{8} w_4, \frac{3}{4} w_5, \frac{1}{8} w_6$\\
$w_5$ größer $\Rightarrow$ $P_5$ bekommt mehr Gewicht $\Rightarrow$ Kurve wird in Richtung $P_5$ gezogen

\section{Hermite-Splines}
\begin{center}
 % 24.2
\end{center}
Kurven, die zwischen zwei gegebenen Endpunkten $P_0$ und $P_3$ interpolieren und dort gegebene Ableitungen $v_0, v_3$
haben. Mehrere solcher Kurven kann mann zusammen setzen, so dass an den Übergängen keine Knicke entstehen.\\[1em]
\textbf{Bezierspline} vom Grad 3:
\begin{center}
 %24.3
\end{center}
\begin{align*}
 v_0 &= X'(0) = 3 \cdot (P_1 - P_0)\\
 v_3 &= X'(3) = 3 \cdot (P_3 - P_2)
\end{align*}

\begin{align*}
 P_1 &= P_0 + \frac{v_0}{3}\\
 P_2 &= P_3 - \frac{v_3}{3}
\end{align*}
\begin{center}
 % 24.4
\end{center}
\begin{center}
 \begin{tabular}{|p{0.98\linewidth}|}
  \hline
  Der Hermite-Spline ist der Bézier-Spline mit den Kontrollpunkten $P_0, P_1, P_2, P_3$\\
  \hline
 \end{tabular}
\end{center}

\Satz Eine Kurve oder Fläche der Klasse $C^k$ ist $k$-mal stetig (\textbf{paramtrische Stetigkeit})
	\begin{itemize}
	 \item $C^0 \dots$ stetig (keine Sprünge)
	 \item $C^1 \dots$ tangentenstetig (keine Knicke)
	 \item $C^2 \dots$ die Krümmung ändert sich stetig.
	\end{itemize}
\begin{center}
 % 23.5
\end{center}
\Defi \textbf{Geometrische Stetigkeit}
\begin{itemize}
 \item $G^1 \dots$ Tangentenrichtung ist überall definiert und ändert sich stetig.
	entsteht beim Zusammensetzen zweier Kurven, die an den Enden parallele Tangenten haben.\\
	Wenn man eine Kurve geeignet umparametrisiert wird die Kurve an dieser Stelle $C_1$-stetig
\end{itemize}
geometrisch ist kein Unterschied zwischen $C_1$ und $G_1$

\section{Parametrische Flächen}
\[X(u,v) \in \mathbb{R}^3 \qquad u, v \in U \subseteq \mathbb{R}^2\]
Kurven $X(t) \in \mathbb{R}^2$ bzw. $X(t) \in \mathbb{R}^3$, $t \in$ Intevall
\paragraph*{Beispiel} Kugel in Kugelkoordinaten
\begin{align*}
 x &= \cos \varphi \cdot \cos \vartheta & \vline && 0 &\le \varphi \le 2\pi\\
 y &= \sin \varphi \cdot \cos \vartheta & \vline && -\frac{\pi}{2} &\le \vartheta \le \frac{\pi}{2}\\
 z &= \sin \vartheta & \vline &&
\end{align*}
\begin{center}
 % 24.7
\end{center}
\begin{align*}
 x(u,v) &= 2 u^2 + 3 v^2 &
 u^2 + v^2 & \le 1 \\
 y(u,v) &= 7uv\\
 y(u,v) &= 3 + 5 u + - v\\
\end{align*}
\begin{center}
  %24.8
\end{center}

Bestimmen der Normalen auf die Fläche im Punkt $X(u,v)$
\begin{align*}
	\frac{\partial X}{\partial u} &= \text{ Die Tangente der Kurve die durch $v = \mathbf{const}$ bestimmt ist}\\
	\frac{\partial X}{\partial v} &= \text{ Die Tangente der Kurve die durch $u = \mathbf{const}$ bestimmt ist}
\end{align*}
$\dfrac{\partial X}{\partial u}$ und $\dfrac{\partial X}{\partial v}$ spannen die Tangetialebene auf (wenn sie nicht
	zufällig parallel sind). Der Normalenvektor $\vec n$ ist darauf senkrecht:
	\[\vec n = \dfrac{\partial X}{\partial u} \times \dfrac{\partial X}{\partial v}\]
\paragraph*{Beispiel} an der Kugel
\begin{align*}
 \frac{\partial X}{\partial u} &= \vthree{\frac{\partial x}{\partial u}}{\frac{\partial y}{\partial u}}
	{\frac{\partial z}{\partial u}} = \vthree{4u}{7v}{5}\\
 \frac{\partial X}{\partial v} &= \vthree{4v}{7u}{-1}\\
 (u,v) &= \left(\frac{1}{2}, \frac{1}{2}\right) \qquad X(u,v) = \vthree{\nicefrac{5}{4}}{\nicefrac{7}{4}}{5}\\
 \vec n &= \vthree{2}{\nicefrac{7}{2}}{5} \times \vthree{2}{\nicefrac{7}{2}}{-1}
\end{align*}

\subsection{Spline-Flächen}
\Defi \textbf{rechteckige $\boldsymbol{a \times b}$-Bezier-Splines}: Gegeben ist ein rechteckiges Feld von
	Kontrollpunkgen $P_{ij}, 0 \le i \le a-1, 0 \le j \le b-1$
	\[\boxed{X(u,v) = \sum\limits_{i=0}^{a-1}\sum\limits_{j=0}^{b-1}
				B_i^a(u)B_j^b(v) P_{ij}}, \qquad 0 \le u \le 1, 0 \le v \le 1\]
	\[\sum\limits_{i=0}^{a-1}\sum\limits_{j=0}^{b-1}
				B_i^a(u)B_j^b =
				\underbrace{\sum\limits_{i=0}^{a-1} B_i^a(u)}_{= 1} \cdot
				\underbrace{\sum\limits_{j=0}^{b-1}B_j^b}_{= 1} = 1\]
	\begin{align*}
	 v = 0 &\Rightarrow \begin{cases}
	                     B_0^b(0) = 1 \\
	                     B_j^b(0) = 1, & j > 0
	                    \end{cases}\\
	 X(u,0) &= \sum\limits_{i=0}^{a-1} B_i^a(u) P_{i0}\\
		&=\ \text{Bezier-Kurve für $P_{00}, P_{10}, ..., P_{a-1,0}$}
	\end{align*}
