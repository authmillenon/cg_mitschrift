\begin{center}
	\definecolor{azure}{rgb}{0.25,0.75,0.25}
 \psset{unit=4cm,xMin=0,xMax=1.5,yMin=0,yMax=1.5,zMin=0,zMax=1.5,Alpha=60,linecolor=azure,Beta=20,nameX=$r$,nameY=$g$,nameZ=$b$}
 \begin{pspicture}(-1,-0.5)(1,1.5)
  \pstThreeDCoor[linecolor=black]
  \pstThreeDBox(0,0,0)(1,0,0)(0,1,0)(0,0,1)
  \pstThreeDLine(0,0,0)(1,1,1)
  \pstThreeDLine[linecolor=red](1,0,0)(1,1,0)(0,1,0)(0,1,1)(0,0,1)(1,0,1)(1,0,0) % (1, g, 0), 0 \le g \le 1 usw...
  \pstThreeDPut(1,1,1){W}
  \pstThreeDPut(0,0,0){K}
  \pstThreeDPut(1,0,0){R}
  \pstThreeDPut(1,1,0){Y}
  \pstThreeDPut(1,0,1){M}
  \pstThreeDPut(0,1,0){G}
  \pstThreeDPut(0,1,1){C}
  \pstThreeDPut(0,0,1){B}
  \pstThreeDDot(0.5,0.5,0.5) % mit Koord beschriften
  \pstThreeDLine(0.5,0.5,0.5)(0,1,0) % mit Koord beschriften
  \pstThreeDDot(0.25,0.75,0.25) % mit Koord beschriften
 \end{pspicture}
\end{center}
Diese "`stärksten Farben"' sind die Farben din mindestens eine Komponente 0 und mindestens eine Komponente 1 haben
(Farben ohne Grau/Weiß/Schwarz-Anteil).

\subsection{Farbsechseck bzw. Farbkreis}
\begin{center}
 % 9.1
\end{center}
\begin{itemize}
 \item Die Punkte auf diesem Sechseck werden häufig durch einen Winkel ($0^\circ-360^\circ$) parametrisiert.
 \item Startpunkt willkürlich (R = $0^\circ$, Y = $60^\circ$, ...)
 \item Dieser Parameter heißt "`Farbton"', "`Unbuntart"' (engl. \textit{hue} (H)). 
\end{itemize}
\begin{align*}
 40^\circ\ \text{entspricht dann } & \frac{1}{3} \cdot \mathrm{R} + \frac{2}{3} \cdot \mathrm{Y} \left(\frac{1}{3} \cdot 0^\circ
	+ \frac{2}{3} \cdot 60^\circ\right)\\
	&= \frac{1}{3} \cdot (1,0,0) + \frac{2}{3} \cdot (1,1,0) = \left(1, \frac{2}{3}, 0\right)
\end{align*}
Wir erhalte ein neues Farbsystem HSV

\subsection{Andere Farbsysteme}
\subsubsection{HSV-System}
\begin{description}
 \item[hue] (Farbton) $0^\circ \le H \le 360^\circ$
 \item[saturation] (Sättigung) $0 \le S \le 1$
 \item[value] ($\approx$ Helligkeit) $0 \le B \le 1$
\end{description}
$V = 1$ sind die Farben auf den drei Deckseiten des Würfels: mindest einer der drei Werte ist 1
\[max(r,g,b) = 1\]
\begin{center}
	\definecolor{azure}{rgb}{0.25,0.75,0.25}
 \psset{unit=4cm,xMin=0,xMax=1.5,yMin=0,yMax=1.5,zMin=0,zMax=1.5,Alpha=60,linecolor=gray,Beta=20,nameX=$r$,nameY=$g$,nameZ=$b$}
 \begin{pspicture}(-1,-0.5)(1,1.5)
  \pstThreeDCoor[linecolor=black]
  \pstThreeDBox(0,0,0)(1,0,0)(0,1,0)(0,0,1)
  \pstThreeDLine(0,0,0)(1,1,1)
  \pstThreeDSquare[fillstyle=vlines](1,0,0)(0,0,1)(0,1,0)
  \pstThreeDSquare[fillstyle=vlines](0,0,1)(1,0,0)(0,1,0)
  \pstThreeDSquare[fillstyle=vlines](0,1,0)(0,0,1)(1,0,0)
  \pstThreeDPut(1,1,1){W}
  \pstThreeDPut(0,0,0){K}
  \pstThreeDPut(1,0,0){R}
  \pstThreeDPut(1,1,0){Y}
  \pstThreeDPut(1,0,1){M}
  \pstThreeDPut(0,1,0){G}
  \pstThreeDPut(0,1,1){C}
  \pstThreeDPut(0,0,1){B}
 \end{pspicture}
\end{center}
Umrechnung HSV $\to$ RGB:
\begin{align*}
 \vthree{r''}{g''}{b''} &\text{sei die reine Farbe, die $H$ entspricht.}\\
 \vthree{r'}{g'}{b'} &= S \vthree{r''}{g''}{b''} + (1-S) \vthree{1}{1}{1}\\
 \text{Ergebnis } \vthree{r}{g}{b} &= V \vthree{r'}{g'}{b'}
\end{align*}
Umrechnung RGB $\to$ HSV
\begin{center}
 % 9.2
\end{center}
\begin{align*}
 V &= \max(r,g,b)\\
 \vthree{r'}{g'}{b'} = \vthree{r}{g}{b} \cdot \frac{1}{V}\\
 S &= 1 - \min(r',g',b')\\
 \rnode{v''}{\vthree{r''}{g''}{b''}} &= \left[\vthree{r'}{g'}{b'}-(1-S)\vthree{1}{1}{1}\right] \cdot \frac{1}{S}
\end{align*}
\rnode{rF}{reine Farbe} $\rightarrow$ Umwandlung in $H$ \nccurve[angleA=45]{->}{rF}{v''}
\begin{center}
 % 9.3
\end{center}
$\vthree{r''}{g''}{b''}$ ist tatsächlich eine reine Farben
\begin{enumerate}
 \item Wir wissen $\max(r',g',b') = 1$ z. B. $r' = 1$
	\[r'' = [1-(1-S)1] \cdot \frac{1}{S} = S \frac{1}{S} = 1\]
 \item Nehmen wir nun an $(r',g',b') = g' = 1-S$
	\[g'' = [g'-(1-S)1] \cdot \frac{1}{S} = [-1 + S + 1 - S] \frac{1}{S} = 0\]
\end{enumerate}
Für $V = 0$ setze $S, H$ beliebig.\\
Für $V \neq 0$, $S = 0$, setze $H$ beliebig.
\begin{center}
 % 9.4
\end{center}
Nachteil:
\begin{description}
 \item $\mathrm{R} = (1,0,0)$
 \item $\mathrm{Y} = (1,1,0)$
 \item $\mathrm{W} = (1,1,1)$
\end{description}
habe ndenselben $V$-Wert.

\subsubsection{HSL-System}
\begin{description}
 \item[hue] (Farbton) $0^\circ \le H \le 360^\circ$
 \item[saturation] (Sättigung) $0 \le S \le 1$
 \item[lightness] (oder "`luminance"') $L = \tfrac{1}{2}$ enthält das reine Farbensechseck und den Graupunkt
	$\left(\tfrac{1}{2},\tfrac{1}{2},\tfrac{1}{2}\right)$
\end{description}
\begin{center}
 % 9.5
\end{center}
Umrechnung HSL $\to$ RGB:
\begin{align*}
 \vthree{r''}{g''}{b''} &\text{sei die reine Farbe, die $H$ entspricht.}\\
 \vthree{r'}{g'}{b'} &= S \vthree{r''}{g''}{b''} + (1-S) \vthree{1}{1}{1}\\
 \text{Für } 0 \le L \le \frac{1}{2}: \vthree{r}{g}{b} &= \vthree{r'}{g'}{b'} \cdot 2L\\
 \text{Für } \frac{1}{2} \le L \le 1: \vthree{r}{g}{b} &= \vthree{r'}{g'}{b'} \cdot (2-2L) +
			\vthree{1}{1}{1} \cdot (2L - 1)
\end{align*}
\begin{center}
 % 9.6+7
\end{center}
$S = 1$ Farben auf allen 6 Seitenflächen des Würfels



\section{Subtraktive Farbmischung (z. B. beim Drucken)}
3 Grundfarben
\begin{description}
 \item $C = $ cyan, $B, G$ wird durchgelassen, $R$ wird absorbiert
 \item $M = $ magenta, $B, R$ wird durchgelassen, $G$ wird absorbiert
 \item $Y = $ gelb, $R, G$ wird durchgelassen, $B$ wird absorbiert
\end{description}
$C + M$ nur Blau bleibt übrig\\
$C + M + Y = $ schwarz

\subsection{CMY-System}
$0 \le C, M, Y \le 1$
\begin{align*}
 C &:= 1 - r\\
 M &:= 1 - g\\
 Y &:= 1 - b
\end{align*}

\subsection{CMYK-System (Vierfarbdruck)}
Zusätzlich $K = $schwarz. (Das Schwarz von $K$ wird dunkler als von $C + M + Y$ oder um Druckfarbe zu sparen.)
\begin{center}
 % 9.8
\end{center}
\begin{align*}
 K' &:= \min(C, M, Y)\\
 C' &:= C - K'\\
 M' &:= M - K'\\
 Y' &:= Y - K'
\end{align*}
(möglichst viel Farbe durch $K$ ersetzen)

\section{Gängige Farbdarstellung heutzutage}
\begin{itemize}
 \item 8 Bit pro Farbkanal $(r,g,b)$ ... 24 Bit pro Bildpunkt\\
	$\Rightarrow 2^{24} = 16\ \text{Mio. Farben}$ (True Color)
 \item 4-Kanal-Darstellung $(r, g, b, \alpha)$
	$\alpha$ ist für Transparenz:
	\begin{itemize}
	 \item $\alpha = 0...$ durchsichtig; Farbe wird vom Hintegrund genommen,
	 \item $\alpha = 1...$ Farbe wird von $(r, g, b)$ bestimmt
	 \item $0 < \alpha < 1...$ teilweise transparent
	\end{itemize}
	$\rightarrow$ 32 bit pro Pixel
\end{itemize}
