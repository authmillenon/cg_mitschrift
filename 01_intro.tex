\chapter{Einführung}
\section{Organisatorisches}
\subsection{Übungsblätter:}
\begin{itemize}
 \item Ausgabe: Mittwoch, Abgabe: Freitag
 \item Abgabe in Zweiergruppen
 \item 60\% der Punkte müssen erreicht werden
 \item min. einmal Vorrechnen
\end{itemize}

\subsection{Programmierung}
\begin{itemize}
 \item Aufgaben in Java gestaltet
 \item mit OpenGL-Interface
 \item auf Nachfrage kann auch C/C++ verwendet werden
\end{itemize}

\section{Übersicht}
\begin{center}
\begin{psmatrix}
 \rnode{GM}{
 \begin{minipage}{3cm}
  \centering
  Geometrisches Modell\\(im Computer)
 \end{minipage}
 } & \rnode{B1}{Bild} & \\
 \rnode{PH}{Punkthaufen} && \rnode{A}{Auge} & \rnode{G}{Gehirn}\\
 \rnode{W}{Wirklichkeit} & \rnode{B2}{Bild}
 \ncline{GM}{B1}
 \ncline{PH}{GM}
 \ncline{W}{PH}\naput[nrot=90]{\textit{\footnotesize Laserscanner}}
 \ncline{W}{B2}\nbput{\textit{\footnotesize Kamera}}
 \ncline{B2}{GM}
 \ncline{B2}{A}
 \ncline{B1}{A}
 \ncline{W}{A}
 \ncline{A}{G}
 \ncbox[linearc=0.3,boxsize=1cm,nodesep=5pt,linecolor=red]{1,1}{1,2}
 \ncbox[linearc=1.5,boxsize=1.5cm,nodesep=10pt,linecolor=green]{1,1}{3,2}
 \ncbox[linearc=0.3,boxsize=1.5cm,nodesep=5pt,linecolor=blue]{1,1}{2,1}
\end{psmatrix}
\end{center}

\begin{itemize}
 \item \textsc{\color{red}Computergrafik}
 \item \textsc{\color{green}Bildbearbeitung / Bilderkennung}
 \item \textsc{\color{blue}Geometrisches Rechnen / Geometrische Modellierung}
\end{itemize}

\subsection{Fahrplan}
\begin{itemize}
 \item Koordinatiensysteme, geometrische Transformationen
 \item Licht und Farben
 \item Rasterung
 \item Beleuchtung und Schattierung
 \item rendering-pipeline: vom Modell bis zum gerasterten Bildbearbeitung
 \item geometrische Modellierung: Kurven, Flächen und Splines
 \item {\color{red}\large Kein Anwendungskurs für OpenGL, JOGL, Javaview etc.!}
\end{itemize}

